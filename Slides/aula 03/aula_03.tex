% Options for packages loaded elsewhere
\PassOptionsToPackage{unicode}{hyperref}
\PassOptionsToPackage{hyphens}{url}
%
\documentclass[
  10pt,
  ignorenonframetext,
]{beamer}
\usepackage{pgfpages}
\setbeamertemplate{caption}[numbered]
\setbeamertemplate{caption label separator}{: }
\setbeamercolor{caption name}{fg=normal text.fg}
\beamertemplatenavigationsymbolsempty
% Prevent slide breaks in the middle of a paragraph
\widowpenalties 1 10000
\raggedbottom
\setbeamertemplate{part page}{
  \centering
  \begin{beamercolorbox}[sep=16pt,center]{part title}
    \usebeamerfont{part title}\insertpart\par
  \end{beamercolorbox}
}
\setbeamertemplate{section page}{
  \centering
  \begin{beamercolorbox}[sep=12pt,center]{part title}
    \usebeamerfont{section title}\insertsection\par
  \end{beamercolorbox}
}
\setbeamertemplate{subsection page}{
  \centering
  \begin{beamercolorbox}[sep=8pt,center]{part title}
    \usebeamerfont{subsection title}\insertsubsection\par
  \end{beamercolorbox}
}
\AtBeginPart{
  \frame{\partpage}
}
\AtBeginSection{
  \ifbibliography
  \else
    \frame{\sectionpage}
  \fi
}
\AtBeginSubsection{
  \frame{\subsectionpage}
}
\usepackage{lmodern}
\usepackage{amssymb,amsmath}
\usepackage{ifxetex,ifluatex}
\ifnum 0\ifxetex 1\fi\ifluatex 1\fi=0 % if pdftex
  \usepackage[T1]{fontenc}
  \usepackage[utf8]{inputenc}
  \usepackage{textcomp} % provide euro and other symbols
\else % if luatex or xetex
  \usepackage{unicode-math}
  \defaultfontfeatures{Scale=MatchLowercase}
  \defaultfontfeatures[\rmfamily]{Ligatures=TeX,Scale=1}
\fi
\usetheme[]{Berkeley}
\usecolortheme{dove}
\usefonttheme{structurebold}
% Use upquote if available, for straight quotes in verbatim environments
\IfFileExists{upquote.sty}{\usepackage{upquote}}{}
\IfFileExists{microtype.sty}{% use microtype if available
  \usepackage[]{microtype}
  \UseMicrotypeSet[protrusion]{basicmath} % disable protrusion for tt fonts
}{}
\makeatletter
\@ifundefined{KOMAClassName}{% if non-KOMA class
  \IfFileExists{parskip.sty}{%
    \usepackage{parskip}
  }{% else
    \setlength{\parindent}{0pt}
    \setlength{\parskip}{6pt plus 2pt minus 1pt}}
}{% if KOMA class
  \KOMAoptions{parskip=half}}
\makeatother
\usepackage{xcolor}
\IfFileExists{xurl.sty}{\usepackage{xurl}}{} % add URL line breaks if available
\IfFileExists{bookmark.sty}{\usepackage{bookmark}}{\usepackage{hyperref}}
\hypersetup{
  pdftitle={Estruturas de dados e manipulação avançada},
  pdfauthor={Frederico Bertholini},
  hidelinks,
  pdfcreator={LaTeX via pandoc}}
\urlstyle{same} % disable monospaced font for URLs
\newif\ifbibliography
\usepackage{color}
\usepackage{fancyvrb}
\newcommand{\VerbBar}{|}
\newcommand{\VERB}{\Verb[commandchars=\\\{\}]}
\DefineVerbatimEnvironment{Highlighting}{Verbatim}{commandchars=\\\{\}}
% Add ',fontsize=\small' for more characters per line
\usepackage{framed}
\definecolor{shadecolor}{RGB}{248,248,248}
\newenvironment{Shaded}{\begin{snugshade}}{\end{snugshade}}
\newcommand{\AlertTok}[1]{\textcolor[rgb]{0.94,0.16,0.16}{#1}}
\newcommand{\AnnotationTok}[1]{\textcolor[rgb]{0.56,0.35,0.01}{\textbf{\textit{#1}}}}
\newcommand{\AttributeTok}[1]{\textcolor[rgb]{0.77,0.63,0.00}{#1}}
\newcommand{\BaseNTok}[1]{\textcolor[rgb]{0.00,0.00,0.81}{#1}}
\newcommand{\BuiltInTok}[1]{#1}
\newcommand{\CharTok}[1]{\textcolor[rgb]{0.31,0.60,0.02}{#1}}
\newcommand{\CommentTok}[1]{\textcolor[rgb]{0.56,0.35,0.01}{\textit{#1}}}
\newcommand{\CommentVarTok}[1]{\textcolor[rgb]{0.56,0.35,0.01}{\textbf{\textit{#1}}}}
\newcommand{\ConstantTok}[1]{\textcolor[rgb]{0.00,0.00,0.00}{#1}}
\newcommand{\ControlFlowTok}[1]{\textcolor[rgb]{0.13,0.29,0.53}{\textbf{#1}}}
\newcommand{\DataTypeTok}[1]{\textcolor[rgb]{0.13,0.29,0.53}{#1}}
\newcommand{\DecValTok}[1]{\textcolor[rgb]{0.00,0.00,0.81}{#1}}
\newcommand{\DocumentationTok}[1]{\textcolor[rgb]{0.56,0.35,0.01}{\textbf{\textit{#1}}}}
\newcommand{\ErrorTok}[1]{\textcolor[rgb]{0.64,0.00,0.00}{\textbf{#1}}}
\newcommand{\ExtensionTok}[1]{#1}
\newcommand{\FloatTok}[1]{\textcolor[rgb]{0.00,0.00,0.81}{#1}}
\newcommand{\FunctionTok}[1]{\textcolor[rgb]{0.00,0.00,0.00}{#1}}
\newcommand{\ImportTok}[1]{#1}
\newcommand{\InformationTok}[1]{\textcolor[rgb]{0.56,0.35,0.01}{\textbf{\textit{#1}}}}
\newcommand{\KeywordTok}[1]{\textcolor[rgb]{0.13,0.29,0.53}{\textbf{#1}}}
\newcommand{\NormalTok}[1]{#1}
\newcommand{\OperatorTok}[1]{\textcolor[rgb]{0.81,0.36,0.00}{\textbf{#1}}}
\newcommand{\OtherTok}[1]{\textcolor[rgb]{0.56,0.35,0.01}{#1}}
\newcommand{\PreprocessorTok}[1]{\textcolor[rgb]{0.56,0.35,0.01}{\textit{#1}}}
\newcommand{\RegionMarkerTok}[1]{#1}
\newcommand{\SpecialCharTok}[1]{\textcolor[rgb]{0.00,0.00,0.00}{#1}}
\newcommand{\SpecialStringTok}[1]{\textcolor[rgb]{0.31,0.60,0.02}{#1}}
\newcommand{\StringTok}[1]{\textcolor[rgb]{0.31,0.60,0.02}{#1}}
\newcommand{\VariableTok}[1]{\textcolor[rgb]{0.00,0.00,0.00}{#1}}
\newcommand{\VerbatimStringTok}[1]{\textcolor[rgb]{0.31,0.60,0.02}{#1}}
\newcommand{\WarningTok}[1]{\textcolor[rgb]{0.56,0.35,0.01}{\textbf{\textit{#1}}}}
\setlength{\emergencystretch}{3em} % prevent overfull lines
\providecommand{\tightlist}{%
  \setlength{\itemsep}{0pt}\setlength{\parskip}{0pt}}
\setcounter{secnumdepth}{5}

\title{Estruturas de dados e manipulação avançada}
\subtitle{Métodos Quantitativos Aplicados à Ciência Política}
\author{Frederico Bertholini}
\date{28.set.2020}

\begin{document}
\frame{\titlepage}

\begin{frame}[allowframebreaks]
  \tableofcontents[hideallsubsections]
\end{frame}
\hypertarget{data.frame}{%
\section{data.frame}\label{data.frame}}

\begin{frame}{data.frame}
\protect\hypertarget{data.frame-1}{}
\begin{itemize}
\item
  Um data.frame é o mesmo que uma tabela do SQL ou uma planilha Excel
\item
  seus dados provavelmente serão importados para um objeto data.frame
\item
  data.frame's são listas especiais em que todos os elementos possuem o
  mesmo comprimento.
\item
  Cada elemento dessa lista pode ser pensado como uma coluna da tabela -
  ou como uma variável. Uso do `\$'
\item
  Seu comprimento representa o número de linhas - ou seja, de
  observações
\end{itemize}
\end{frame}

\begin{frame}[fragile]{data.frame}
\protect\hypertarget{data.frame-2}{}
\begin{itemize}
\tightlist
\item
  Como data.frames's são listas, suas colunas podem ser de classes
  diferentes. Essa é a grande diferença entre data.frame's e matrizes.
\end{itemize}

Funções úteis:

\begin{Shaded}
\begin{Highlighting}[]
\KeywordTok{head}\NormalTok{() }\CommentTok{\# Mostra as primeiras 6 linhas.}
\KeywordTok{tail}\NormalTok{() }\CommentTok{\# Mostra as últimas 6 linhas.}
\KeywordTok{dim}\NormalTok{() }\CommentTok{\# Número de linhas e de colunas.}
\KeywordTok{names}\NormalTok{() }\CommentTok{\# Os nomes das colunas (variáveis).}
\KeywordTok{str}\NormalTok{() }\CommentTok{\# Estrutura do data.frame. Mostra, entre outras coisas, as \textgreater{}classes de cada coluna.}
\KeywordTok{cbind}\NormalTok{() }\CommentTok{\# Acopla duas tabelas lado a lado.}
\KeywordTok{rbind}\NormalTok{() }\CommentTok{\# Empilha duas tabelas.}
\end{Highlighting}
\end{Shaded}
\end{frame}

\hypertarget{pacotes}{%
\section{Pacotes}\label{pacotes}}

\begin{frame}{O que são pacotes}
\protect\hypertarget{o-que-suxe3o-pacotes}{}
\begin{itemize}
\tightlist
\item
  O R possui diversas funções já instaladas dentro da sua programação
\item
  Exemplos são sum(), length(), class(), c()
\item
  Outras, porém, devem ser instaladas para que possam ser utilizadas
  pelos usuários
\item
  A maneira com a qual instalamos novas funções, não definidas
  anteriormente no software, é através de pacotes
\item
  Pacotes concentram diversas funções para diversas demandas\\
  Importação de dados;\\
  Organização de banco de dados;\\
  Análises estatísticas específicas;\\
  Gráficos diferenciados;
\end{itemize}
\end{frame}

\begin{frame}[fragile]{O que são pacotes}
\protect\hypertarget{o-que-suxe3o-pacotes-1}{}
\begin{itemize}
\tightlist
\item
  A instalação de qualquer pacote pode ser feita por dentro do R
\item
  Para isso, porém, é preciso primeiro conexão com a internet, já que o
  R busca o novo pacote no repositório de pacotes
\item
  A função para instalar pacote, portanto, é install.packages()
\item
  O nome da nova função deve vir, primeiramente entre parenteses
\item
  Podemos começar instalando o pacote para importação de bases de dados:
  foreign
\end{itemize}

\begin{Shaded}
\begin{Highlighting}[]
\KeywordTok{install.packages}\NormalTok{(}\StringTok{"foreign"}\NormalTok{)}
\end{Highlighting}
\end{Shaded}

\begin{itemize}
\tightlist
\item
  Após alguns segundos, e algumas mensagens no console, a instalação
  será efetivada
\end{itemize}
\end{frame}

\begin{frame}[fragile]{Ativar pacotes}
\protect\hypertarget{ativar-pacotes}{}
\begin{itemize}
\tightlist
\item
  Cada pacote, inclusive o foreign, tem uma documentação disponível na
  internet
\item
  Nessa documentação estão disponíveis as funções que o pacote possui,
  além do nome do seu criador
\item
  As função não ficam disponíveis assim que o pacote termina a
  instalação
\item
  Para ativar as funções do pacote, é preciso utilizar a função
  library()
\end{itemize}

\begin{Shaded}
\begin{Highlighting}[]
\KeywordTok{library}\NormalTok{(foreign)}
\end{Highlighting}
\end{Shaded}

\begin{itemize}
\tightlist
\item
  Repare, que uma vez instalado, o nome do pacote não precisa mais estar
  entre aspas
\end{itemize}
\end{frame}

\begin{frame}[fragile]{Ativar pacotes}
\protect\hypertarget{ativar-pacotes-1}{}
\begin{itemize}
\tightlist
\item
  Portanto, para começarmos o procedimento de instalação do pacote,
  seguimos os passos a seguir:\\
  Caso não esteja instalado, instalar o pacote através da função
  install.packages();\\
  Para ativar o pacote, utilizar a função library() sem as aspas no nome
  do pacote instalado
\end{itemize}

\begin{Shaded}
\begin{Highlighting}[]
\KeywordTok{install.packages}\NormalTok{(}\StringTok{"foreign"}\NormalTok{)}

\KeywordTok{library}\NormalTok{(foreign)}
\end{Highlighting}
\end{Shaded}

\begin{itemize}
\tightlist
\item
  Uma vez instalado o pacote, não é preciso instalar mais a não ser que
  você reinstale o R
\end{itemize}
\end{frame}

\hypertarget{importauxe7uxe3o-de-dados}{%
\section{Importação de dados}\label{importauxe7uxe3o-de-dados}}

\begin{frame}[fragile]{Passo a passo}
\protect\hypertarget{passo-a-passo}{}
\begin{itemize}
\tightlist
\item
  A importação é uma das tarefas que demandam mais atenção no R
\item
  É preciso ter um conhecimento prévio de como sua base externa está
  constituída
\item
  Outra informação importante é a extensão do arquivo da base
\item
  Primeiramente, a informação que deve ser dada ao software é onde está
  a base - diretório de trabalho
\item
  A função necessária é setwd() que define o diretório da sua seção no R
\item
  Dentro da função, iremos inserir o local do arquivo\\
  Em caso de Windows, inverta as barras ou duplique;\\
  Não se esqueça das aspas;
\end{itemize}

\begin{Shaded}
\begin{Highlighting}[]
\KeywordTok{setwd}\NormalTok{(}\StringTok{"/Volumes/Macintosh HD/MQCP\_IPOL\_2020/Slides/aula 03"}\NormalTok{)}
\end{Highlighting}
\end{Shaded}
\end{frame}

\begin{frame}[fragile]{Passo a passo}
\protect\hypertarget{passo-a-passo-1}{}
\begin{itemize}
\tightlist
\item
  Esse diretório definido significa que as bases e os gráficos
  produzidos serão enviados para essa pasta
\item
  Finalmente, vamos importar as bases de dados
\item
  Primeiro, vamos importar a base de extensão txt com o nome baserm
\item
  Não é preciso de pacote para esse procedimento
\end{itemize}

\begin{Shaded}
\begin{Highlighting}[]
\NormalTok{lines \textless{}{-}}\StringTok{ }\KeywordTok{readLines}\NormalTok{(}\StringTok{"dados/baserm.txt"}\NormalTok{)}

\NormalTok{baserm \textless{}{-}}\StringTok{ }\KeywordTok{read.table}\NormalTok{(}\DataTypeTok{text =}\NormalTok{ lines, }\DataTypeTok{sep =} \StringTok{\textquotesingle{}}\CharTok{\textbackslash{}t}\StringTok{\textquotesingle{}}\NormalTok{)}
\end{Highlighting}
\end{Shaded}

\begin{itemize}
\tightlist
\item
  Repare que definimos a base dentro das aspas e com a extensão
\item
  Na segunda linha, o primeiro argumento é o texto, o segundo argumento
  trata de como os dados estão separados, geralmente txt vem separado
  assim
\end{itemize}
\end{frame}

\begin{frame}[fragile]{Passo a passo}
\protect\hypertarget{passo-a-passo-2}{}
\begin{itemize}
\tightlist
\item
  No pacote foreign, a forma mais genérica de importação da base é o
  read.table()
\item
  Entretanto, o pacote apresenta uma série de especialidades, a depender
  da extensão em questão
\item
  Para CSV, vimos que tem a read.csv(). Já para dta, base de origem do
  stata, temos a função read.dta()
\item
  O pacote foreign não possui a extensão xlsx e xls, extensão muito
  encontrada e comum entre as bases de dados disponíveis
\item
  Para isso, vamos instalar um novo pacote readxl
\end{itemize}

\begin{Shaded}
\begin{Highlighting}[]
\KeywordTok{install.packages}\NormalTok{(}\StringTok{"readxl"}\NormalTok{)}
\end{Highlighting}
\end{Shaded}

\begin{itemize}
\tightlist
\item
  Esse pacote disponibiliza as funções read\_xls() e read\_xlsx()
\end{itemize}
\end{frame}

\begin{frame}[fragile]{Passo a passo}
\protect\hypertarget{passo-a-passo-3}{}
\begin{itemize}
\tightlist
\item
  Vamos ativar as funções disponíveis no pacote readxl com a função
  library()
\end{itemize}

\begin{Shaded}
\begin{Highlighting}[]
\KeywordTok{library}\NormalTok{(readxl)}
\end{Highlighting}
\end{Shaded}

\begin{itemize}
\tightlist
\item
  Vamos importar a base controle\_cgu\_municípios.xlsx
\end{itemize}

\begin{Shaded}
\begin{Highlighting}[]
\NormalTok{cgu \textless{}{-}}\StringTok{ }
\KeywordTok{read\_xlsx}\NormalTok{(}\StringTok{"dados/controle\_cgu\_municípios.xlsx"}\NormalTok{)}
\end{Highlighting}
\end{Shaded}

\begin{itemize}
\tightlist
\item
  Repare que acessamos apenas a primeira página da base
\item
  Para acessarmos a segunda páginas, utilizamos o argumento sheet=2
\end{itemize}

\begin{Shaded}
\begin{Highlighting}[]
\NormalTok{cgu \textless{}{-}}\StringTok{ }
\KeywordTok{read\_xlsx}\NormalTok{(}\StringTok{"dados/controle\_cgu\_municípios.xlsx"}\NormalTok{, }
\DataTypeTok{sheet =} \DecValTok{2}\NormalTok{)}
\end{Highlighting}
\end{Shaded}

\begin{itemize}
\tightlist
\item
  Alguns sinais de alerta surgem, porém não se trata de erro
\end{itemize}
\end{frame}

\begin{frame}{Importação por pacote}
\protect\hypertarget{importauxe7uxe3o-por-pacote}{}
\begin{itemize}
\tightlist
\item
  Por fim, podemos importar dados através de pacotes
\item
  Após ativar o pacote ``ElectionsBR'', a função legend\_fed() fica
  disponível para a importação
\item
  Como toda a função, ou quase todas, possui argumentos importantes:

  \begin{itemize}
  \tightlist
  \item
    `year =' se refere ao ano de extração
  \item
    `uf =' se refere à UF
  \end{itemize}
\end{itemize}
\end{frame}

\begin{frame}[fragile]{Importação}
\protect\hypertarget{importauxe7uxe3o}{}
\begin{itemize}
\tightlist
\item
  Vamos começar importando dados de coalizões pré-eleitorais
  (coligações) do DF em 2018, nos retornando um objeto em `tbl\_df' e
  data frame
\end{itemize}

\begin{Shaded}
\begin{Highlighting}[]
\KeywordTok{library}\NormalTok{(electionsBR)}
\end{Highlighting}
\end{Shaded}

\begin{verbatim}
## 
## To cite electionsBR in publications, use: citation('electionsBR')
\end{verbatim}

\begin{verbatim}
## To learn more, visit: http://electionsbr.com
\end{verbatim}

\begin{Shaded}
\begin{Highlighting}[]
\NormalTok{leg\_df\_}\DecValTok{2018}\NormalTok{ \textless{}{-}}\StringTok{ }\KeywordTok{legend\_fed}\NormalTok{(}\DataTypeTok{year =} \DecValTok{2018}\NormalTok{,}\DataTypeTok{uf=}\StringTok{"DF"}\NormalTok{)}
\end{Highlighting}
\end{Shaded}

\begin{verbatim}
## Processing the data...
\end{verbatim}

\begin{verbatim}
## Warning: `as.tbl()` is deprecated as of dplyr 1.0.0.
## Please use `tibble::as_tibble()` instead.
## This warning is displayed once every 8 hours.
## Call `lifecycle::last_warnings()` to see where this warning was generated.
\end{verbatim}

\begin{verbatim}
## Done.
\end{verbatim}

\begin{verbatim}
  - Provavelmente, demorará alguns segundos, até minutos, para a importação
  - Além disso, essa técnica demanda acesso à internet
\end{verbatim}
\end{frame}

\begin{frame}[fragile]{Visualizando a base}
\protect\hypertarget{visualizando-a-base}{}
\begin{itemize}
\tightlist
\item
  Primeira coisa importante de se informar é a classe desses objetos
\item
  Temos 4 objetos: baserm, cgu, educacao e pnad2018
\end{itemize}

\begin{Shaded}
\begin{Highlighting}[]
\KeywordTok{class}\NormalTok{(baserm)}
\end{Highlighting}
\end{Shaded}

\begin{verbatim}
## [1] "data.frame"
\end{verbatim}

\begin{Shaded}
\begin{Highlighting}[]
\KeywordTok{class}\NormalTok{(cgu)}
\end{Highlighting}
\end{Shaded}

\begin{verbatim}
## [1] "tbl_df"     "tbl"        "data.frame"
\end{verbatim}

\begin{Shaded}
\begin{Highlighting}[]
\KeywordTok{class}\NormalTok{(leg\_df\_}\DecValTok{2018}\NormalTok{)}
\end{Highlighting}
\end{Shaded}

\begin{verbatim}
## [1] "tbl_df"     "tbl"        "data.frame"
\end{verbatim}
\end{frame}

\begin{frame}[fragile]{Visualizando a base}
\protect\hypertarget{visualizando-a-base-1}{}
\begin{itemize}
\tightlist
\item
  Uma visão completa da base é o comando View()
\item
  Entretanto, cuidado, dependendo do tamanho da base, podemos travar o
  software
\end{itemize}

\begin{Shaded}
\begin{Highlighting}[]
\KeywordTok{View}\NormalTok{(baserm)}
\end{Highlighting}
\end{Shaded}

\begin{itemize}
\tightlist
\item
  Repare no V maiúsculo, lembre-se que o R é bastante sensível na sua
  linguagem
\item
  O View() abre uma nova aba com a base no formato de grade
\item
  Podemos, assim, visualizar a base de dados na forma mais intuitiva
\end{itemize}
\end{frame}

\begin{frame}[fragile]{Visualizando a base}
\protect\hypertarget{visualizando-a-base-2}{}
\begin{itemize}
\tightlist
\item
  Porém, para bases como a pnad2018, por exemplo, sabemos que é grande
  demais para sua visualização ser feita através do View()
\item
  Algumas funções podem nos ajudar nessa tarefa
\item
  A primeira é o dim, que as dimensões da base
\end{itemize}

\begin{Shaded}
\begin{Highlighting}[]
\KeywordTok{dim}\NormalTok{(leg\_df\_}\DecValTok{2018}\NormalTok{)}
\end{Highlighting}
\end{Shaded}

\begin{verbatim}
## [1] 251  23
\end{verbatim}

\begin{itemize}
\tightlist
\item
  O primeiro valor sempre retrata o número de linhas, ou observações,
  enquanto o segundo valor apresenta o número de colunas, ou variáveis
\item
  A função ncol() e length() também indicam quantas colunas, ou
  variáveis estão presentes na base
\end{itemize}
\end{frame}

\begin{frame}[fragile]{Visualizando a base}
\protect\hypertarget{visualizando-a-base-3}{}
\begin{itemize}
\tightlist
\item
  Outra função importante na visualização de bases de dados é a lista de
  nomes
\item
  A função names() descreve as variáveis presentes na base
\item
  Isso facilita no momento de selecionar as variáveis que entrarão na
  análise de vocês
\end{itemize}

\begin{Shaded}
\begin{Highlighting}[]
\KeywordTok{names}\NormalTok{(cgu)}
\end{Highlighting}
\end{Shaded}

\begin{verbatim}
##  [1] "COD.IBGE7"               "REGIÃO"                 
##  [3] "UF"                      "PORTE"                  
##  [5] "MUNICÍPIO"               "falha"                  
##  [7] "tempo_falha_02"          "tempo_falha_01"         
##  [9] "reicidência_falha"       "ano_eleitoral"          
## [11] "PERCENT_ganhador_2000"   "PERCENT_ganhador_2004"  
## [13] "PERCENT_ganhador_2008"   "PERCENT_ganhador_2012"  
## [15] "PERCENT_ganhador_médio"  "competição_pol_alta"    
## [17] "Ideologia1"              "Ideologia2"             
## [19] "Ideologia3"              "ideologia 4"            
## [21] "ideologia_media"         "IDHM_2000"              
## [23] "IDHM_2010"               "IDHM_media"             
## [25] "tm 2000"                 "tm 2004"                
## [27] "tm 2008"                 "tm2012"                 
## [29] "tamanho setor media"     "NEP1"                   
## [31] "NEP2"                    "NEP3"                   
## [33] "NEP4"                    "NEP_medio"              
## [35] "Pibpercapita2000"        "Pibpercapita2004"       
## [37] "Pibpercapita2008"        "Pibpercapita2012"       
## [39] "Pibpercapita_medio"      "TaxadeUrbanização2000"  
## [41] "TaxadeUrbanização2004"   "TaxadeUrbanização2008"  
## [43] "TaxadeUrbanização_media" "Arrec_Prop2000"         
## [45] "Arrec_Prop2004"          "Arrec_Prop2008"         
## [47] "Arrec_Prop2012"          "Arrec_Prop_media"       
## [49] "POP"                     "LN_POP"                 
## [51] "N_NE"                    "Viznho"                 
## [53] "sul"                     "centro_Oeste"           
## [55] "Sudeste"                 "Nordeste"
\end{verbatim}
\end{frame}

\begin{frame}[fragile]{Visualizando a base}
\protect\hypertarget{visualizando-a-base-4}{}
\begin{itemize}
\tightlist
\item
  Outra função possível é o str()
\item
  Essa função apresenta o nome das variáveis, a classe de cada uma delas
  e os primeiros valores
\end{itemize}

\begin{Shaded}
\begin{Highlighting}[]
\KeywordTok{str}\NormalTok{(leg\_df\_}\DecValTok{2018}\NormalTok{)}
\end{Highlighting}
\end{Shaded}

\begin{verbatim}
## tibble [251 x 23] (S3: tbl_df/tbl/data.frame)
##  $ DATA_GERACAO        : chr [1:251] "23/09/2020" "23/09/2020" "23/09/2020" "23/09/2020" ...
##  $ HORA_GERACAO        : 'hms' num [1:251] 17:31:43 17:31:43 17:31:43 17:31:43 ...
##   ..- attr(*, "units")= chr "secs"
##  $ ANO_ELEICAO         : num [1:251] 2018 2018 2018 2018 2018 ...
##  $ COD_TIPO_ELEICAO    : num [1:251] 2 2 2 2 2 2 2 2 2 2 ...
##  $ NM_TIPO_ELEICAO     : chr [1:251] "ELEIÇÃO ORDINÁRIA" "ELEIÇÃO ORDINÁRIA" "ELEIÇÃO ORDINÁRIA" "ELEIÇÃO ORDINÁRIA" ...
##  $ NUM_TURNO           : num [1:251] 1 1 1 1 1 1 1 1 1 1 ...
##  $ COD_ELEICAO         : num [1:251] 297 297 297 297 297 297 297 297 297 297 ...
##  $ DESCRICAO_ELEICAO   : chr [1:251] "Eleições Gerais Estaduais 2018" "Eleições Gerais Estaduais 2018" "Eleições Gerais Estaduais 2018" "Eleições Gerais Estaduais 2018" ...
##  $ DATA_ELEICAO        : chr [1:251] "07/10/2018" "07/10/2018" "07/10/2018" "07/10/2018" ...
##  $ SIGLA_UF            : chr [1:251] "DF" "DF" "DF" "DF" ...
##  $ SIGLA_UE            : chr [1:251] "DF" "DF" "DF" "DF" ...
##  $ NOME_MUNICIPIO      : chr [1:251] "DISTRITO FEDERAL" "DISTRITO FEDERAL" "DISTRITO FEDERAL" "DISTRITO FEDERAL" ...
##  $ CODIGO_CARGO        : num [1:251] 4 6 5 6 9 8 5 4 3 9 ...
##  $ DESCRICAO_CARGO     : chr [1:251] "VICE-GOVERNADOR" "DEPUTADO FEDERAL" "SENADOR" "DEPUTADO FEDERAL" ...
##  $ TIPO_LEGENDA        : chr [1:251] "COLIGACAO" "COLIGACAO" "COLIGACAO" "COLIGACAO" ...
##  $ NUMERO_PARTIDO      : num [1:251] 12 35 18 20 22 65 12 29 21 17 ...
##  $ SIGLA_PARTIDO       : chr [1:251] "PDT" "PMB" "REDE" "PSC" ...
##  $ NOME_PARTIDO        : chr [1:251] "PARTIDO DEMOCRÁTICO TRABALHISTA" "PARTIDO DA MULHER BRASILEIRA" "REDE SUSTENTABILIDADE" "PARTIDO SOCIAL CRISTÃO" ...
##  $ SEQUENCIAL_COLIGACAO: num [1:251] 7e+10 7e+10 7e+10 7e+10 7e+10 ...
##  $ NOME_COLIGACAO      : chr [1:251] "Brasília de Mãos Limpas" "RENOVAR DF 2" "Brasília de Mãos Limpas" "UNIDOS PELO DF 1" ...
##  $ COMPOSICAO_COLIGACAO: chr [1:251] "PSB / PV / PC do B / PDT / REDE" "PROS / PMB / PMN" "PSB / PV / PC do B / PDT / REDE" "PRB / PODE / PPS / SOLIDARIEDADE / PSC / PSD" ...
##  $ NA                  : chr [1:251] "D" "D" "D" "D" ...
##  $ NA                  : chr [1:251] "DEFERIDO" "DEFERIDO" "DEFERIDO" "DEFERIDO" ...
##  - attr(*, ".internal.selfref")=<externalptr>
\end{verbatim}
\end{frame}

\begin{frame}[fragile]{Visualizando a base}
\protect\hypertarget{visualizando-a-base-5}{}
\begin{itemize}
\tightlist
\item
  Finalmente, a função head() e tail()
\item
  A primeira função apresenta os primeiros valores de uma base de dados
\end{itemize}

\begin{Shaded}
\begin{Highlighting}[]
\KeywordTok{head}\NormalTok{(baserm,}\DecValTok{2}\NormalTok{)}
\end{Highlighting}
\end{Shaded}

\begin{verbatim}
##   sigla cod.ibge  estado anoeleitoral1990 anoeleitoral1991 anoeleitoral1992
## 1    AC       12    Acre                1                0                0
## 2    AL       27 Alagoas                1                0                0
##   anoeleitoral1993 anoeleitoral1994 anoeleitoral1995 anoeleitoral1996
## 1                0                1                0                0
## 2                0                1                0                0
##   anoeleitoral1997 anoeleitoral1998 anoeleitoral1999 anoeleitoral2000
## 1                0                1                0                0
## 2                0                1                0                0
##   anoeleitoral2001 anoeleitoral2002 anoeleitoral2003 anoeleitoral2004
## 1                0                1                0                0
## 2                0                1                0                0
##   anoeleitoral2005 anoeleitoral2006 anoeleitoral2007 anoeleitoral2008
## 1                0                1                0                0
## 2                0                1                0                0
##   anoeleitoral2009 anoeleitoral2010 anoeleitoral2011 anoeleitoral2012
## 1                0                1                0                0
## 2                0                1                0                0
##   anoeleitoral2013 anoeleitoral2014 anoeleitoral2015 anoeleitoral2016
## 1                0                1                0                0
## 2                0                1                0                0
##   anoeleitoral2017 vitoria1990.1 vitoria1994.1 vitoria1998.1 vitoria2002.1
## 1                0         28.64         46.80         57.70         46.81
## 2                0         49.76         79.39         58.09         52.93
##   vitoria2006.1 vitoria2010.1 vitoria2014.1 vitoria2017.1 derrota1990.1
## 1         53.05         50.51         49.73            NA         28.34
## 2         55.85         39.58         52.16            NA         44.65
##   derrota1994.1 derrota1998.1 derrota2002.1 derrota2006.1 derrota2010.1
## 1         27.30         26.30         33.64         35.12         49.18
## 2          9.97         38.95         40.17         30.51         29.16
##   derrota2014.1 derrota2017.1 marge1990 marge1994 marge1998 marge2002 marge2006
## 1         30.10            NA      0.30     19.50     31.40     13.17     17.93
## 2         33.92            NA      5.11     69.42     19.14     12.76     25.34
##   marge2010 marge2014 marge2017 vitoria1990.2 vitoria1994.2 vitoria1998.2
## 1      1.33     19.63        NA         54.61         53.66            NA
## 2     10.42     18.24        NA         65.97            NA            NA
##   vitoria2002.2 vitoria2006.2 vitoria2010.2 vitoria2014.2 vitoria2017.2
## 1            NA            NA            NA         51.29            NA
## 2            NA            NA         52.74            NA            NA
##   derrota1990.2 derrota1994.2 derrota1998.2 derrota2002.2 derrota2006.2
## 1         45.39         46.34            NA            NA            NA
## 2         34.03            NA            NA            NA            NA
##   derrota2010.2 derrota2014.2 derrota2017.2 segunt.1990 segunt.1994 segunt.1998
## 1            NA         48.71            NA           1           1           0
## 2         47.26            NA            NA           1           0           0
##   segunt.2002 segunt.2006 segunt.2010 segunt.2014 segunt.2017
## 1           0           0           0           1           0
## 2           0           0           1           0           0
##                        colig.venc1990             colig.venc1994
## 1                         PDS/PFL/PRN                 PPR/PP/PFL
## 2 PSC/PMDB/PFL/PSDB/PDT/PTB/PMN/PTdoB PMDB/PTB/PSDB/PL/PSD/PCdoB
##                                            colig.venc1998
## 1       PT/PSDB/PCdoB/PDT/PSB/PPS/PV/PMN/PTB/PL/PSL/PTdoB
## 2 PSB/PT/PDT/PCdoB/PPS/PTdoB/PRONA/PRP/PV/PTN/PSN/PMN/PST
##                          colig.venc2002              colig.venc2006
## 1         PT/PL/PCdoB/PV/PMN/PSDC/PTdoB PT/PP/PL/PSB/PCdoB/PMN/PRTB
## 2 PSB/PST/PAN/PHS/PSC/PTC/PSDC/PV/PRONA         PSDB/PMDB/PPS/PTdoB
##                                                    colig.venc2010
## 1 PT/PP/PRB/PDT/PTN/PR/PSDC/PHS/PTC/PSB/PV/PRP/PCB/PTB/PSTU/PCdoB
## 2                                         PP/PSC/PPS/DEM/PSB/PSDB
##                                                   colig.venc2014 colig.venc2017
## 1 PT/PDT/PCdoB/PSB/PTB/PCB/PSL/PRB/PSDC/PTN/PPL/PROS/PRP/PHS/PEN           <NA>
## 2                PMDB/PT/PDT/PTB/PCdoB/PSD/PSC/PROS/PV/PTdoB/PHS           <NA>
##   cent.venc1990 cent.venc1994 cent.venc1998 cent.venc2002 cent.venc2006
## 1           PDS           PPR            PT            PT            PT
## 2           PSC          PMDB           PSB           PSB          PSDB
##   cent.venc2010 cent.venc2014 cent.venc2017         colig.derr1990
## 1            PT            PT          <NA>       PT/PDT/PCB/PCdoB
## 2          PSDB          PMDB          <NA> PRN/PDC/PL/PTR/PST/PRP
##           colig.derr1994        colig.derr1998        colig.derr2002
## 1          PMDB/PSDB/PDT               PFL/PPB PMDB/PSDB/PFL/PPB/PST
## 2 PP/PPR/PFL/PSC/PRN/PMN PTB/PSDB/PFL/PGT/PMDB  PRTB/PTB/PPS/PPB/PFL
##         colig.derr2006                      colig.derr2010
## 1 PPS/PMDB/PDT/PTC/PHS PSDB/PMDB/DEM/PSL/PSC/PPS/PMN/PTdoB
## 2    PTB/PMN/PFL/PV/PP PDT/PT/PMDB/PR/PSDC/PRP/PCdoB/PTdoB
##                             colig.derr2014 colig.derr2017 cent.derr1990
## 1 PSDB/PMDB/PP/PSC/PSD/PR/PPS/PTdoB/SD/PTC           <NA>            PT
## 2        PP/PSB/DEM/PR/PPS/SD/PSL/PSDC/PRP           <NA>           PRN
##   cent.derr1994 cent.derr1998 cent.derr2002 cent.derr2006 cent.derr2010
## 1          PMDB           PFL          PMDB           PPS          PSDB
## 2            PP           PTB          PRTB           PTB           PDT
##   cent.derr2014 cent.derr2017      pib1990      pib1991      pib1992
## 1          PSDB          <NA> 8,5046506371 8,7232716989 8,5758084005
## 2            PP          <NA> 6,1245303676 6,1700559571 6,3226355082
##        pib1993      pib1994      pib1995      pib1996      pib1997      pib1998
## 1 8,5919435813 8,4737417079 8,5777487009 8,5137013716 8,2219297645 8,3880379051
## 2 6,0055725931  6,362685085  6,181753266 6,1250000647 6,3801014987  6,447201319
##        pib1999      pib2000      pib2001      pib2002      pib2003      pib2004
## 1  8,304872692 8,4004032965 8,4772487716 8,6074182777 8,7019837683 9,2998229259
## 2 6,2677884172 6,3133529976 6,3528049829 6,2872324787  6,178023009 6,3815555297
##        pib2005      pib2006       pib2007       pib2008       pib2009
## 1  9,396913016 9,6611772592 10,7836698517 11,1098707418 11,0597301472
## 2 6,6081828111 6,8190719202  7,1293278871  7,2077456254  7,2927619283
##         pib2010       pib2011 presidente_ass1990 presidente_ass1994
## 1 11,5674100019 11,0150052537                PDS                PPB
## 2  7,8742100008  8,4879911718               <NA>               <NA>
##   presidente_ass1998 presidente_ass2002 presidente_ass2006 presidente_ass2010
## 1                PPB                PMN                PPR                PEN
## 2               <NA>               <NA>               <NA>               <NA>
##   presidente_ass2014 alinhamento1990 alinhamento1994 alinhamento1998
## 1                 PT               1               0               0
## 2               PMDB              NA              NA              NA
##   alinhamento2002 alinhamento2006 alinhamento2010 alinhamento2014  nep1994
## 1               0               0               0               1 2.027088
## 2              NA              NA              NA               1 4.300454
##         tampub1990       tampub1991       tampub1992       tampub1993
## 1 611411,513586656  536496,02078722 608472,003975918 628730,202973373
## 2 1313572,46142367 1194027,27279608 1028453,06094235 1347342,39625056
##         tampub1994       tampub1995       tampub1996       tampub1997
## 1 759921,483162687 702943,384175584 641316,605942853  732135,48830176
## 2 1312023,68646294 1383956,67383895 1801731,17072279 1759142,06212779
##         tampub1998       tampub1999       tampub2000       tampub2001
## 1 791582,505357824 769742,256433218 709631,314458073 696843,832285717
## 2 1828943,13052009 1775942,55060186 1773535,41143224 1794722,67176703
##         tampub2002       tampub2003       tampub2004       tampub2005
## 1 792129,607169136 790293,863518309 812197,028281624 865132,591797147
## 2 1784143,23672935 1759563,77863612 1795456,15261078 1912450,83760713
##         tampub2006       tampub2007       tampub2008       tampub2009
## 1 932586,565877418 1014670,56483533 1070900,44301039  1100656,5650149
## 2 2075462,54896424 2262606,57966598 2454839,30043114 2559887,84486059
\end{verbatim}

\begin{itemize}
\tightlist
\item
  O segundo argumento serve para indicar quantas linhas serão
  apresentadas
\item
  Tail(), por outro lado, apresenta os últimos valores de uma base de
  dados
\end{itemize}

\begin{Shaded}
\begin{Highlighting}[]
\KeywordTok{tail}\NormalTok{(baserm,}\DecValTok{4}\NormalTok{)}
\end{Highlighting}
\end{Shaded}

\begin{verbatim}
##    sigla cod.ibge         estado anoeleitoral1990 anoeleitoral1991
## 24    SC       42 Santa Catarina                1                0
## 25    SE       28        Sergipe                1                0
## 26    SP       35   S<e3>o Paulo                1                0
## 27    TO       17      Tocantins                1                0
##    anoeleitoral1992 anoeleitoral1993 anoeleitoral1994 anoeleitoral1995
## 24                0                0                1                0
## 25                0                0                1                0
## 26                0                0                1                0
## 27                0                0                1                0
##    anoeleitoral1996 anoeleitoral1997 anoeleitoral1998 anoeleitoral1999
## 24                0                0                1                0
## 25                0                0                1                0
## 26                0                0                1                0
## 27                0                0                1                0
##    anoeleitoral2000 anoeleitoral2001 anoeleitoral2002 anoeleitoral2003
## 24                0                0                1                0
## 25                0                0                1                0
## 26                0                0                1                0
## 27                0                0                1                0
##    anoeleitoral2004 anoeleitoral2005 anoeleitoral2006 anoeleitoral2007
## 24                0                0                1                0
## 25                0                0                1                0
## 26                0                0                1                0
## 27                0                0                1                0
##    anoeleitoral2008 anoeleitoral2009 anoeleitoral2010 anoeleitoral2011
## 24                0                0                1                0
## 25                0                0                1                0
## 26                0                0                1                0
## 27                0                0                1                0
##    anoeleitoral2012 anoeleitoral2013 anoeleitoral2014 anoeleitoral2015
## 24                0                0                1                0
## 25                0                0                1                0
## 26                0                0                1                0
## 27                0                0                1                0
##    anoeleitoral2016 anoeleitoral2017 vitoria1990.1 vitoria1994.1 vitoria1998.1
## 24                0                0         50.42         45.86         58.92
## 25                0                0         73.74         47.61         40.17
## 26                0                0         43.50         46.84         32.21
## 27                0                0         49.12         58.73         61.65
##    vitoria2002.1 vitoria2006.1 vitoria2010.1 vitoria2014.1 vitoria2017.1
## 24         39.86         48.90         52.71         51.36            NA
## 25         43.63         52.46         52.08         53.52            NA
## 26         38.28         57.93         50.63         57.31            NA
## 27         60.60         51.49         50.52         51.30            NA
##    derrota1990.1 derrota1994.1 derrota1998.1 derrota2002.1 derrota2006.1
## 24         30.07         34.00         23.12         30.07         32.77
## 25         25.07         47.40         38.70         28.43         45.02
## 26         28.17         22.23         22.95         32.45         31.68
## 27         37.33         36.57         33.17         33.74         46.84
##    derrota2010.1 derrota2014.1 derrota2017.1 marge1990 marge1994 marge1998
## 24         24.91         29.90            NA     20.35     11.86     35.80
## 25         45.19         41.37            NA     48.67      0.21      1.47
## 26         35.23         21.53            NA     15.33     24.61      9.26
## 27         49.48         44.72            NA     11.79     22.16     28.48
##    marge2002 marge2006 marge2010 marge2014 marge2017 vitoria1990.2
## 24      9.79     16.13     27.80     21.46        NA            NA
## 25     15.20      7.44      6.89     12.15        NA            NA
## 26      5.83     26.25     15.40     35.78        NA         51.77
## 27     26.86      4.65      1.04      6.58        NA         55.94
##    vitoria1994.2 vitoria1998.2 vitoria2002.2 vitoria2006.2 vitoria2010.2
## 24         50.80            NA         50.34         52.71            NA
## 25         51.68         54.39         55.00            NA            NA
## 26         56.12         55.37         58.64            NA            NA
## 27            NA            NA            NA            NA            NA
##    vitoria2014.2 vitoria2017.2 derrota1990.2 derrota1994.2 derrota1998.2
## 24            NA            NA            NA         49.20            NA
## 25            NA            NA            NA         48.32         45.61
## 26            NA            NA         48.23         43.88         44.63
## 27            NA            NA         44.06            NA            NA
##    derrota2002.2 derrota2006.2 derrota2010.2 derrota2014.2 derrota2017.2
## 24         49.66         47.29            NA            NA            NA
## 25         45.00            NA            NA            NA            NA
## 26         41.36            NA            NA            NA            NA
## 27            NA            NA            NA            NA            NA
##    segunt.1990 segunt.1994 segunt.1998 segunt.2002 segunt.2006 segunt.2010
## 24           0           1           0           1           1           0
## 25           0           1           1           1           0           0
## 26           1           1           1           1           0           0
## 27           1           0           0           0           0           0
##    segunt.2014 segunt.2017                       colig.venc1990
## 24           0           0              PFL/PDS /PDC/PTB/PL/PSC
## 25           0           0 PFL/PMDB/PDS/PSDB/PRN/PL/PDC/PCN/PST
## 26           0           0                      PMDB/PFL/PL/PSD
## 27           0           0              PMDB/PDT/PSDB/PRN/PCdoB
##                                    colig.venc1994
## 24  PMDB/PTRB/PMN/PSD/PV/PRP/PCdoB/PDT/PPS/PT/PSB
## 25 PSDB/PFL/PMDB/PPR/PTB/PPS/PL/PSD/PSC/PRP/PTdoB
## 26                                       PSDB/PFL
## 27                          PPR/PFL/PP/PTB/PV/PMN
##                                                        colig.venc1998
## 24            PPB/PFL/PSDB/PTB/PL/PRN/PTdoB/PGT/PRTB/PSDC/PSL/PST/PAN
## 25                                    PSDB/PMDB/PPB/PL/PSC/PPS/PV/PMN
## 26                                                       PSDB/PTB/PSD
## 27 PFL/PPB/PSDB/PTB/PSC/PL/PDT/PSB/PV/PRP/PST/PTN/PSDC/PGT/PRTB/PTdoB
##                                       colig.venc2002
## 24                                         PMDB/PSDB
## 25                 PFL/PPB/PPS/PDT/PSD/PST/PHS/PTdoB
## 26                                      PSDB/PFL/PSD
## 27 PFL/PPB/PSDB/PSD/PSL/PST/PAN/PRTB/PRP/PRONA/PTdoB
##                          colig.venc2006
## 24 PMDB/PAN/PFL/PHS/PPS/PRTB/PSDB/PTdoB
## 25             PT/PSB/PMDB/PL/PTB/PCdoB
## 26                     PSDB/PFL/PTB/PPS
## 27                         PMDB/PPS/PFL
##                                       colig.venc2010
## 24             DEM/PMDB/PSDB/PPS/PTB/PSC/PSL/PTC/PRP
## 25          PT/PMDB/PDT/PSB/PRB/PSL/PSC/PR/PTC/PCdoB
## 26                     PSDB/DEM/PMDB/PSC/PPS/PHS/PMN
## 27 PSDB/DEM/PTB/PR/PSC/PV/PTN/PRB/PRTB/PMN/PTC/PTdoB
##                                                colig.venc2014 colig.venc2017
## 24         PSD/PMDB/DEM/PRB/PR/PSC/PSDC/PROS/PV/PDT/PCdoB/PTB           <NA>
## 25           PMDB/PSB/PT/PRB/PSD/PCdoB/PDT/PROS/PRTB/PSDC/PRP           <NA>
## 26 PSDB/PSB/DEM/PRB/SD/PSC/PPS/PTC/PTN/PSL/PMN/PEN/PSDC/PTdoB           <NA>
## 27                                             PMDB/PV/PT/PSD           <NA>
##    cent.venc1990 cent.venc1994 cent.venc1998 cent.venc2002 cent.venc2006
## 24           PFL          PMDB           PPB          PMDB          PMDB
## 25           PFL          PSDB          PSDB           PFL            PT
## 26          PMDB          PSDB          PSDB          PSDB          PSDB
## 27          PMDB           PPR           PFL           PFL          PMDB
##    cent.venc2010 cent.venc2014 cent.venc2017              colig.derr1990
## 24           DEM           PSD          <NA>                        <NA>
## 25            PT          PMDB          <NA> PT/PSB/PCB/PCdoB/PDT/PMN/PV
## 26          PSDB          PSDB          <NA>             PDS/PTB/PDC/PRN
## 27          PSDB          PMDB          <NA>  PDC/PFL/PL/PSC/PST/PTR/PMN
##                            colig.derr1994                    colig.derr1998
## 24             PPR/PP/PTB/PSC/PL/PFL/PSDB                              <NA>
## 25                PDT/PMN/PP/PT/PSB/PCdoB PFL/PTB/PSD/PRN/PRP/PTN/PAN/PTdoB
## 26                             PDT/PV/PRP            PPB/PFL/PL/PRN/PSL/PST
## 27 PMDB/PSDB/PDT/PL/PSC/PSD/PRN/PSB/PCdoB                PMDB/PCdoB/PPS/PSD
##                        colig.derr2002                           colig.derr2006
## 24 PPB/PFL/PSL/PST/PDT/PTB/PRTB/PTdoB                          PP/PMN/PRONA/PV
## 25                PT/PL/PCdoB/PCB/PMN PFL/PSDB/PP/PPS/PSC/PV/PTN/PAN/PHS/PTdoB
## 26                   PT/PCdoB/PCB/PMN                     PT/PCdoB/PL/PRB/PRTB
## 27                               <NA>          PSDB/PP/PTB/PSC/PL/PSB/PV/PTdoB
##                                   colig.derr2010
## 24                                  PP/PDT/PTdoB
## 25                                          <NA>
## 26   PT/PDT/PCdoB/PR/PSDC/PTN/PRB/PRP/PRTB/PTdoB
## 27 PMDB/PP/PT/PDT/PPS/PSB/PSL/PSDC/PHS/PRP/PCdoB
##                                                        colig.derr2014
## 24                  PSDB/PP/PSB/PSL/PTN/PPS/PRTB/PHS/PTC/PEN/PTdoB/SD
## 25         PSC/PSDB/DEM/PTB/PPS/PR/SD/PTC/PEN/PV/PHS/PMN/PSL/PP/PTdoB
## 26                                               PMDB/PDT/PSD/PP/PROS
## 27 SD/PDT/PSDB/PP/PTB/DEM/PPS/PSB/PRB/PSL/PSC/PR/PRTB/PHS/PTC/PRP/PEN
##    colig.derr2017 cent.derr1990 cent.derr1994 cent.derr1998 cent.derr2002
## 24           <NA>          PMDB           PPR          PMDB           PPB
## 25           <NA>            PT           PDT           PFL            PT
## 26           <NA>           PDS           PDT           PPB            PT
## 27           <NA>           PDC          PMDB          PMDB          PMDB
##    cent.derr2006 cent.derr2010 cent.derr2014 cent.derr2017       pib1990
## 24            PP            PP          PSDB          <NA> 17,2942659152
## 25           PFL          PRTB           PSC          <NA>   8,492623942
## 26            PT            PT          PMDB          <NA> 22,9467064865
## 27          PSDB          PMDB            SD          <NA>  7,4802792384
##          pib1991       pib1992       pib1993       pib1994       pib1995
## 24 17,4862919047 17,7565110336 18,4320880928 18,7359294484 19,5010579924
## 25  8,4435004265  8,2226447498  8,4335197127  8,4825655399  8,4215892959
## 26 22,4485135313 21,5229856589 22,4841773891 23,4447255754 24,1643222911
## 27  7,6122181897  7,6062331607  7,5925648284  7,6169350505  7,7713175348
##          pib1996       pib1997       pib1998       pib1999       pib2000
## 24 19,8858255241 20,4634529228 20,1427853348 20,0822904535 20,5757218084
## 25   8,561773119  9,0157557207  8,8628500033  8,6218728818  8,7672435729
## 26 24,1031639625 24,5496938133 23,8962432492 23,1100251143 23,7092725642
## 27  7,7757470707  7,9080208851  8,0789021631  8,0857119891  8,1320619577
##          pib2001       pib2002       pib2003       pib2004       pib2005
## 24 20,8844623857 20,9616339366 20,8384159784 22,0522553398 22,0431784035
## 25  8,6820460686  8,8717633988  8,9504386847  9,3804812523   9,746751866
## 26 23,4228446292 23,3234417859 22,8757550908 23,8912154431 24,3598569518
## 27  8,0138560417  8,0758971422  8,7326442958  9,3678485905  9,7264417171
##          pib2006       pib2007       pib2008       pib2009       pib2010
## 24 22,2606141093   23,96452476 23,9180809893 23,6424543172 24,3984200027
## 25  9,9776921472 10,9347879797 10,8817029713 11,2505972278 11,5724400015
## 26 24,9467930565 27,6206370268 28,4034940399 27,9288107975 30,2431700034
## 27   9,829354866 11,0237822544 11,3575263384 11,6840034913  12,461670002
##          pib2011 presidente_ass1990 presidente_ass1994 presidente_ass1998
## 24 25,0174683914                PDS                PDS               PSDB
## 25 11,7197545373                PFL               <NA>                PFL
## 26 30,3351441728               PMDB               PSDB               PSDB
## 27 12,0513847615               <NA>               <NA>               <NA>
##    presidente_ass2002 presidente_ass2006 presidente_ass2010 presidente_ass2014
## 24                 PT               <NA>                PSD               PMDB
## 25                PFL               PSDB                PSC               PMDB
## 26               PSDB               PSDB                PSB               PSDB
## 27               <NA>               <NA>               <NA>               <NA>
##    alinhamento1990 alinhamento1994 alinhamento1998 alinhamento2002
## 24               0               0               0               0
## 25               1              NA               0               1
## 26               1               1               1               1
## 27              NA              NA              NA              NA
##    alinhamento2006 alinhamento2010 alinhamento2014  nep1994       tampub1990
## 24              NA               0               0 1.852039 3864440,51885973
## 25               0               0               1 1.540130 971919,356653973
## 26               1               0               1 1.108644 32030505,5565219
## 27              NA              NA              NA 3.267424 336618,073532113
##          tampub1991       tampub1992       tampub1993       tampub1994
## 24 3044398,00952559 2878119,75130662 3233298,35490963 3883578,63480444
## 25 1033326,59668615 960209,465362672 1023101,23672086 1093757,75386495
## 26 29318383,7079537 26083676,9601677 29504411,1559908 30518738,9985881
## 27  395704,33220278 299986,954758008 458680,363556348 545919,364528239
##          tampub1995       tampub1996       tampub1997       tampub1998
## 24 4239392,34190224  4329228,8310874 3933863,15598997 4118723,21824177
## 25 1359761,69411054 1369064,65139731  1287005,6603136 1371625,49255682
## 26 32451245,0300568 36454840,1901522 38684167,2518864 41904103,8133446
## 27 576491,552242776  695778,42898056 706966,447184362 793937,877438079
##          tampub1999       tampub2000       tampub2001       tampub2002
## 24 4001104,51547176 4351501,64784291 4069839,68215204 4487510,19538674
## 25 1332917,63824243 1399417,02637658 1347882,17013854 1593092,32922432
## 26 40720661,2869789 34337301,0888346 32533001,4269696  31570213,574016
## 27 771778,031620068 618568,172025916 591323,579673879 942548,599020474
##          tampub2003       tampub2004       tampub2005       tampub2006
## 24 4583523,25202327 4580035,05220288 4997253,66029641 5253204,99279084
## 25 1541196,53447207 1550972,17264608 1694634,93724624 1960397,99369845
## 26 31326075,7776173  30883214,726262 32673443,6775475 34138254,7949099
## 27 980592,141717068 1038045,61036685 1109369,15573646 1265587,84651625
##          tampub2007       tampub2008       tampub2009
## 24  5672797,5807212 6149095,66638515 6262294,59770163
## 25 2082847,29585588 2232124,70578603 2266779,22969508
## 26 37750820,6501719  39947491,184932 42897556,2016075
## 27 1387668,85772441  1464607,2921382 1575495,46613015
\end{verbatim}
\end{frame}

\hypertarget{o-universo-tidyverse}{%
\section{O Universo tidyverse}\label{o-universo-tidyverse}}

\begin{frame}{Manifesto tidyverse}
\protect\hypertarget{manifesto-tidyverse}{}
O tidyverse, também chamado por muitos de hadleyverse, é um conjunto de
pacotes que, por compartilharem esses princípios do manifesto tidy,
podem ser utilizados naturalmente em conjunto. Pode-se dizer que existe
o R antes do tidyverse e o R depois do tidyverse.

Os princípios fundamentais do tidyverse são:

\begin{itemize}
\item
  Reutilizar estruturas de dados existentes.
\item
  Organizar funções simples usando o pipe.
\item
  Aderir à programação funcional.
\item
  Projetado para ser usado por seres humanos.
\end{itemize}
\end{frame}

\begin{frame}{Manifesto tidy}
\protect\hypertarget{manifesto-tidy}{}
\begin{itemize}
\item
  Tidy Tools Manifesto
  \url{https://cran.r-project.org/web/packages/tidyverse/vignettes/manifesto.html}
\item
  Tidy data vignette
  \url{https://cran.r-project.org/web/packages/tidyr/vignettes/tidy-data.html}
\item
  Tidy Data paper \url{http://vita.had.co.nz/papers/tidy-data.pdf}
\item
  Conjunto de pacotes \url{https://www.tidyverse.org/packages/}
\end{itemize}
\end{frame}

\begin{frame}[fragile]{Usando o pipe - O operador \%\textgreater\%}
\protect\hypertarget{usando-o-pipe---o-operador}{}
O operador \%\textgreater\% (pipe) foi uma das grandes revoluções
recentes do R, tornando a leitura de códigos mais lógica, fácil e
compreensível.

\begin{Shaded}
\begin{Highlighting}[]
\KeywordTok{library}\NormalTok{(tidyverse)}
\KeywordTok{library}\NormalTok{(magrittr)}
\end{Highlighting}
\end{Shaded}
\end{frame}

\begin{frame}[fragile]{Ideia}
\protect\hypertarget{ideia}{}
A ideia do operador \%\textgreater\% (pipe) é bem simples: usar o valor
resultante da expressão do lado esquerdo como primeiro argumento da
função do lado direito.

\begin{itemize}
\tightlist
\item
  As duas linhas abaixo são equivalentes.
\end{itemize}

\begin{Shaded}
\begin{Highlighting}[]
\KeywordTok{f}\NormalTok{(x, y)}
\end{Highlighting}
\end{Shaded}

\begin{Shaded}
\begin{Highlighting}[]
\NormalTok{x }\OperatorTok{\%\textgreater{}\%}\StringTok{ }\KeywordTok{f}\NormalTok{(y)}
\end{Highlighting}
\end{Shaded}
\end{frame}

\begin{frame}[fragile]{E se aumentarmos o código?}
\protect\hypertarget{e-se-aumentarmos-o-cuxf3digo}{}
Vamos calcular a raiz quadrada da soma dos valores de 1 a 4.

Primeiro, sem o pipe.

\begin{Shaded}
\begin{Highlighting}[]
\KeywordTok{sqrt}\NormalTok{(}\KeywordTok{sum}\NormalTok{(x))}
\end{Highlighting}
\end{Shaded}

\begin{verbatim}
## [1] 3.162278
\end{verbatim}

Agora com o pipe.

\begin{Shaded}
\begin{Highlighting}[]
\NormalTok{x }\OperatorTok{\%\textgreater{}\%}\StringTok{ }
\StringTok{  }\NormalTok{sum }\OperatorTok{\%\textgreater{}\%}\StringTok{ }
\StringTok{  }\NormalTok{sqrt}
\end{Highlighting}
\end{Shaded}

\begin{verbatim}
## [1] 3.162278
\end{verbatim}
\end{frame}

\begin{frame}{E se realmente tivermos muitas funções aninhadas?}
\protect\hypertarget{e-se-realmente-tivermos-muitas-funuxe7uxf5es-aninhadas}{}
A utilização do pipe transforma um código confuso e difícil de ser lido
em algo \emph{simples e intuitivo}.
\end{frame}

\begin{frame}[fragile]{Receita de bolo - sem pipe}
\protect\hypertarget{receita-de-bolo---sem-pipe}{}
Tente entender o que é preciso fazer.

\begin{Shaded}
\begin{Highlighting}[]
\KeywordTok{esfrie}\NormalTok{(}
  \KeywordTok{asse}\NormalTok{(}
    \KeywordTok{coloque}\NormalTok{(}
      \KeywordTok{bata}\NormalTok{(}
        \KeywordTok{acrescente}\NormalTok{(}
          \KeywordTok{recipiente}\NormalTok{(}\KeywordTok{rep}\NormalTok{(}\StringTok{"farinha"}\NormalTok{, }\DecValTok{2}\NormalTok{), }\StringTok{"água"}\NormalTok{, }
                     \StringTok{"fermento"}\NormalTok{, }\StringTok{"leite"}\NormalTok{, }\StringTok{"óleo"}\NormalTok{), }
          \StringTok{"farinha"}\NormalTok{, até =}\StringTok{ "macio"}\NormalTok{), }
\NormalTok{        duração =}\StringTok{ "3min"}\NormalTok{), }
      \DataTypeTok{lugar =} \StringTok{"forma"}\NormalTok{, }\DataTypeTok{tipo =} \StringTok{"grande"}\NormalTok{, }
      \DataTypeTok{untada =} \OtherTok{TRUE}\NormalTok{), duração =}\StringTok{ "50min"}\NormalTok{), }
  \StringTok{"geladeira"}\NormalTok{, }\StringTok{"20min"}\NormalTok{)}
\end{Highlighting}
\end{Shaded}
\end{frame}

\begin{frame}[fragile]{Receita de bolo - com pipe}
\protect\hypertarget{receita-de-bolo---com-pipe}{}
Desistiu? Agora veja como fica escrevendo com o
\texttt{\%\textgreater{}\%}:

\begin{Shaded}
\begin{Highlighting}[]
\KeywordTok{recipiente}\NormalTok{(}\KeywordTok{rep}\NormalTok{(}\StringTok{"farinha"}\NormalTok{, }\DecValTok{2}\NormalTok{), }\StringTok{"água"}\NormalTok{, }\StringTok{"fermento"}\NormalTok{, }\StringTok{"leite"}\NormalTok{, }\StringTok{"óleo"}\NormalTok{) }\OperatorTok{\%\textgreater{}\%}
\StringTok{  }\KeywordTok{acrescente}\NormalTok{(}\StringTok{"farinha"}\NormalTok{, até =}\StringTok{ "macio"}\NormalTok{) }\OperatorTok{\%\textgreater{}\%}
\StringTok{  }\KeywordTok{bata}\NormalTok{(duração =}\StringTok{ "3min"}\NormalTok{) }\OperatorTok{\%\textgreater{}\%}
\StringTok{  }\KeywordTok{coloque}\NormalTok{(}\DataTypeTok{lugar =} \StringTok{"forma"}\NormalTok{, }\DataTypeTok{tipo =} \StringTok{"grande"}\NormalTok{, }\DataTypeTok{untada =} \OtherTok{TRUE}\NormalTok{) }\OperatorTok{\%\textgreater{}\%}
\StringTok{  }\KeywordTok{asse}\NormalTok{(duração =}\StringTok{ "50min"}\NormalTok{) }\OperatorTok{\%\textgreater{}\%}
\StringTok{  }\KeywordTok{esfrie}\NormalTok{(}\StringTok{"geladeira"}\NormalTok{, }\StringTok{"20min"}\NormalTok{)}
\end{Highlighting}
\end{Shaded}
\end{frame}

\hypertarget{importauxe7uxe3o-no-tidyverse}{%
\section{Importação no tidyverse}\label{importauxe7uxe3o-no-tidyverse}}

\begin{frame}[fragile]{Importação com \texttt{readr}, \texttt{readxl},
\texttt{haven} e \texttt{DBI}}
\protect\hypertarget{importauxe7uxe3o-com-readr-readxl-haven-e-dbi}{}
No tidyverse, geralmente

\begin{itemize}
\tightlist
\item
  Funções \texttt{read\_\textless{}formato\textgreater{}} servem para
  ler um arquivo no formato \texttt{\textless{}formato\textgreater{}}
\item
  Funções \texttt{write\_\textless{}formato\textgreater{}} servem para
  escrever num arquivo com o formato
  \texttt{\textless{}formato\textgreater{}}
\end{itemize}
\end{frame}

\begin{frame}[fragile]{Arquivos de texto}
\protect\hypertarget{arquivos-de-texto}{}
\begin{itemize}
\item
  \texttt{csv}, \texttt{tsv}, \texttt{txt}, \ldots{}
\item
  Para esses aqui, usar o pacote \texttt{readr}
\item
  Você também pode experimentar o \texttt{data.table::fread}
\end{itemize}
\end{frame}

\begin{frame}[fragile]{Arquivos binários}
\protect\hypertarget{arquivos-binuxe1rios}{}
\begin{itemize}
\item
  \texttt{.RData}, \texttt{.rds}, \texttt{.feather}, \texttt{.fst}
\item
  \texttt{.dta} (Stata), \texttt{.sas7bdat} (SAS), \texttt{.sav} (SPSS)
\item
  Ler com \texttt{readr}, \texttt{haven}, \texttt{feather},
  \texttt{fst}.
\end{itemize}
\end{frame}

\begin{frame}[fragile]{Bancos de dados}
\protect\hypertarget{bancos-de-dados}{}
\begin{itemize}
\item
  \texttt{MySQL}, \texttt{SQL\ Server}, \texttt{PostgreSQL},
  \texttt{SQLite}, \ldots{}
\item
  \texttt{Spark}, \texttt{MongoDB}, \texttt{Hive}, \ldots{}
\item
  Utilizar pacotes \texttt{DBI} e \texttt{odbc}
\end{itemize}
\end{frame}

\begin{frame}[fragile]{Tidy data e janitor}
\protect\hypertarget{tidy-data-e-janitor}{}
Veremos mais à frente, mas janitor::clean\_names() é uma ferramenta tidy

\begin{Shaded}
\begin{Highlighting}[]
\KeywordTok{library}\NormalTok{(janitor)}
\end{Highlighting}
\end{Shaded}

\begin{verbatim}
## 
## Attaching package: 'janitor'
\end{verbatim}

\begin{verbatim}
## The following objects are masked from 'package:stats':
## 
##     chisq.test, fisher.test
\end{verbatim}
\end{frame}

\hypertarget{pacotes-dplyr-e-tidyr}{%
\section{\texorpdfstring{Pacotes \texttt{dplyr} e
\texttt{tidyr}}{Pacotes dplyr e tidyr}}\label{pacotes-dplyr-e-tidyr}}

\begin{frame}[fragile]{Conjunto de dados}
\protect\hypertarget{conjunto-de-dados}{}
Vamos trabalhar com a base \texttt{decisoes}, que contém decisões do
Tribunal de Justiça de São Paulo

\begin{Shaded}
\begin{Highlighting}[]
\NormalTok{decisoes \textless{}{-}}\StringTok{ }\KeywordTok{read\_rds}\NormalTok{(}\StringTok{"dados/decisoes.rds"}\NormalTok{)}
\KeywordTok{glimpse}\NormalTok{(decisoes)}
\end{Highlighting}
\end{Shaded}

\begin{verbatim}
## Rows: 11,731
## Columns: 9
## $ `ID Decisão`     <chr> "11094999", "11093733", "11093677", "11093270", "1...
## $ n_processo       <chr> "0057003-20.2017.8.26.0000", "0052762-03.2017.8.26...
## $ `Classe/Assunto` <chr> "Habeas Corpus / Homicídio Simples", "Habeas Corpu...
## $ Município        <chr> "Cosmópolis", "São Paulo", "Ribeirão Preto", "Araç...
## $ Câmara           <chr> "3ª Câmara de Direito Criminal", "3ª Câmara de Dir...
## $ `Data decisão`   <chr> "19/12/2017", "19/12/2017", "19/12/2017", "14/12/2...
## $ `Data registro`  <chr> "19/12/2017", "19/12/2017", "19/12/2017", "19/12/2...
## $ Juiz             <chr> "Luiz Antonio Cardoso", "Luiz Antonio Cardoso", "L...
## $ `txt decisão`    <chr> NA, NA, NA, "Execução Penal –  Comutação de Penas ...
\end{verbatim}

\begin{Shaded}
\begin{Highlighting}[]
\NormalTok{decisoes \textless{}{-}}\StringTok{ }\KeywordTok{read\_rds}\NormalTok{(}\StringTok{"dados/decisoes.rds"}\NormalTok{) }\OperatorTok{\%\textgreater{}\%}
\StringTok{  }\NormalTok{janitor}\OperatorTok{::}\KeywordTok{clean\_names}\NormalTok{() }\CommentTok{\# com dois pontos eu não preciso usar library}
\KeywordTok{glimpse}\NormalTok{(decisoes)}
\end{Highlighting}
\end{Shaded}

\begin{verbatim}
## Rows: 11,731
## Columns: 9
## $ id_decisao     <chr> "11094999", "11093733", "11093677", "11093270", "110...
## $ n_processo     <chr> "0057003-20.2017.8.26.0000", "0052762-03.2017.8.26.0...
## $ classe_assunto <chr> "Habeas Corpus / Homicídio Simples", "Habeas Corpus ...
## $ municipio      <chr> "Cosmópolis", "São Paulo", "Ribeirão Preto", "Araçat...
## $ camara         <chr> "3ª Câmara de Direito Criminal", "3ª Câmara de Direi...
## $ data_decisao   <chr> "19/12/2017", "19/12/2017", "19/12/2017", "14/12/201...
## $ data_registro  <chr> "19/12/2017", "19/12/2017", "19/12/2017", "19/12/201...
## $ juiz           <chr> "Luiz Antonio Cardoso", "Luiz Antonio Cardoso", "Lui...
## $ txt_decisao    <chr> NA, NA, NA, "Execução Penal –  Comutação de Penas – ...
\end{verbatim}
\end{frame}

\begin{frame}[fragile]{Características do \texttt{dplyr}}
\protect\hypertarget{caracteruxedsticas-do-dplyr}{}
\begin{itemize}
\item
  A utilização é facilitada com o emprego do operador
  \texttt{\%\textgreater{}\%}
\item
  No primeiro argumento colocamos o \texttt{data.frame} ou o
  \texttt{tibble}, e nos outros argumentos colocamos o que queremos
  fazer.
\end{itemize}
\end{frame}

\begin{frame}[fragile]{As cinco funções principais do \texttt{dplyr}}
\protect\hypertarget{as-cinco-funuxe7uxf5es-principais-do-dplyr}{}
\begin{itemize}
\item
  \texttt{select}: selecionar colunas
\item
  \texttt{filter}: filtrar linhas
\item
  \texttt{mutate}: criar colunas
\item
  \texttt{summarise}: sumarizar colunas
\item
  \texttt{arrange}: ordenar linhas
\end{itemize}
\end{frame}

\hypertarget{select}{%
\section{select}\label{select}}

\begin{frame}[fragile]{\texttt{select}}
\protect\hypertarget{select-1}{}
\begin{itemize}
\tightlist
\item
  Utilizar \texttt{starts\_with(x)}, \texttt{contains(x)},
  \texttt{matches(x)}, \texttt{one\_of(x)}, etc.
\item
  Possível colocar nomes, índices, e intervalos de variáveis com
  \texttt{:}.
\end{itemize}
\end{frame}

\begin{frame}[fragile]{Em ação}
\protect\hypertarget{em-auxe7uxe3o}{}
\begin{Shaded}
\begin{Highlighting}[]
\NormalTok{decisoes }\OperatorTok{\%\textgreater{}\%}\StringTok{ }
\StringTok{  }\KeywordTok{select}\NormalTok{(id\_decisao, n\_processo, municipio, juiz)}
\end{Highlighting}
\end{Shaded}

\begin{verbatim}
## # A tibble: 11,731 x 4
##    id_decisao n_processo                municipio           juiz                
##    <chr>      <chr>                     <chr>               <chr>               
##  1 11094999   0057003-20.2017.8.26.0000 Cosmópolis          Luiz Antonio Cardoso
##  2 11093733   0052762-03.2017.8.26.0000 São Paulo           Luiz Antonio Cardoso
##  3 11093677   0055169-79.2017.8.26.0000 Ribeirão Preto      Luiz Antonio Cardoso
##  4 11093270   9000580-82.2017.8.26.0032 Araçatuba           Grassi Neto         
##  5 11093374   0052938-79.2017.8.26.0000 São Paulo           Grassi Neto         
##  6 11093320   9000723-79.2017.8.26.0482 Presidente Prudente Grassi Neto         
##  7 11091506   0003276-86.2015.8.26.0075 Bertioga            Grassi Neto         
##  8 11093326   9000298-11.2017.8.26.0625 Taubaté             Grassi Neto         
##  9 11092475   0004653-39.2015.8.26.0028 Aparecida           Grassi Neto         
## 10 11093773   2221930-66.2017.8.26.0000 Jandira             Luiz Antonio Cardoso
## # ... with 11,721 more rows
\end{verbatim}
\end{frame}

\begin{frame}[fragile]{Em ação}
\protect\hypertarget{em-auxe7uxe3o-1}{}
\begin{Shaded}
\begin{Highlighting}[]
\NormalTok{decisoes }\OperatorTok{\%\textgreater{}\%}\StringTok{ }
\StringTok{  }\KeywordTok{select}\NormalTok{(classe\_assunto}\OperatorTok{:}\NormalTok{id\_decisao, juiz)}
\end{Highlighting}
\end{Shaded}

\begin{verbatim}
## # A tibble: 11,731 x 4
##    classe_assunto                      n_processo        id_decisao juiz        
##    <chr>                               <chr>             <chr>      <chr>       
##  1 Habeas Corpus / Homicídio Simples   0057003-20.2017.~ 11094999   Luiz Antoni~
##  2 Habeas Corpus / Roubo               0052762-03.2017.~ 11093733   Luiz Antoni~
##  3 Habeas Corpus / DIREITO PENAL       0055169-79.2017.~ 11093677   Luiz Antoni~
##  4 Agravo de Execução Penal / Pena Pr~ 9000580-82.2017.~ 11093270   Grassi Neto 
##  5 Mandado de Segurança / Crimes do S~ 0052938-79.2017.~ 11093374   Grassi Neto 
##  6 Agravo de Execução Penal / Pena Pr~ 9000723-79.2017.~ 11093320   Grassi Neto 
##  7 Apelação / Tráfico de Drogas e Con~ 0003276-86.2015.~ 11091506   Grassi Neto 
##  8 Agravo de Execução Penal / Livrame~ 9000298-11.2017.~ 11093326   Grassi Neto 
##  9 Apelação / Tráfico de Drogas e Con~ 0004653-39.2015.~ 11092475   Grassi Neto 
## 10 Habeas Corpus / Furto Qualificado   2221930-66.2017.~ 11093773   Luiz Antoni~
## # ... with 11,721 more rows
\end{verbatim}
\end{frame}

\begin{frame}[fragile]{Em ação}
\protect\hypertarget{em-auxe7uxe3o-2}{}
\begin{Shaded}
\begin{Highlighting}[]
\NormalTok{decisoes }\OperatorTok{\%\textgreater{}\%}\StringTok{ }
\StringTok{  }\KeywordTok{select}\NormalTok{(id\_decisao, }\KeywordTok{starts\_with}\NormalTok{(}\StringTok{\textquotesingle{}data\_\textquotesingle{}}\NormalTok{))}
\end{Highlighting}
\end{Shaded}

\begin{verbatim}
## # A tibble: 11,731 x 3
##    id_decisao data_decisao data_registro
##    <chr>      <chr>        <chr>        
##  1 11094999   19/12/2017   19/12/2017   
##  2 11093733   19/12/2017   19/12/2017   
##  3 11093677   19/12/2017   19/12/2017   
##  4 11093270   14/12/2017   19/12/2017   
##  5 11093374   14/12/2017   19/12/2017   
##  6 11093320   14/12/2017   19/12/2017   
##  7 11091506   14/12/2017   19/12/2017   
##  8 11093326   14/12/2017   19/12/2017   
##  9 11092475   14/12/2017   19/12/2017   
## 10 11093773   19/12/2017   19/12/2017   
## # ... with 11,721 more rows
\end{verbatim}
\end{frame}

\begin{frame}[fragile]{Operações}
\protect\hypertarget{operauxe7uxf5es}{}
\begin{itemize}
\tightlist
\item
  selecione as colunas que acabam com ``cisao''.
\end{itemize}

\begin{Shaded}
\begin{Highlighting}[]
\NormalTok{decisoes }\OperatorTok{\%\textgreater{}\%}\StringTok{ }
\StringTok{  }\KeywordTok{select}\NormalTok{(}\KeywordTok{ends\_with}\NormalTok{(}\StringTok{"cisao"}\NormalTok{))}
\end{Highlighting}
\end{Shaded}

\begin{verbatim}
## # A tibble: 11,731 x 3
##    id_decisao data_decisao txt_decisao                                          
##    <chr>      <chr>        <chr>                                                
##  1 11094999   19/12/2017    <NA>                                                
##  2 11093733   19/12/2017    <NA>                                                
##  3 11093677   19/12/2017    <NA>                                                
##  4 11093270   14/12/2017   "Execução Penal –  Comutação de Penas –  Decreto n. ~
##  5 11093374   14/12/2017   "Mandado de segurança – Impetração por agente despro~
##  6 11093320   14/12/2017   "Execução Penal – Apuração de falta grave em procedi~
##  7 11091506   14/12/2017   "Tráfico de entorpecentes – Agente que traz consigo ~
##  8 11093326   14/12/2017   "Execução Penal – Pedido de livramento condicional d~
##  9 11092475   14/12/2017   "Tráfico de entorpecentes –  Agente que tem em depós~
## 10 11093773   19/12/2017    <NA>                                                
## # ... with 11,721 more rows
\end{verbatim}
\end{frame}

\begin{frame}[fragile]{Operações}
\protect\hypertarget{operauxe7uxf5es-1}{}
\begin{itemize}
\tightlist
\item
  tire as colunas de texto = `txt\_decisao' e classe/assunto =
  `classe\_assunto'.

  \begin{itemize}
  \tightlist
  \item
    Dica: veja os exemplos de \texttt{?select} em
    \texttt{Drop\ variables\ ...}
  \end{itemize}
\end{itemize}

\begin{Shaded}
\begin{Highlighting}[]
\NormalTok{decisoes }\OperatorTok{\%\textgreater{}\%}\StringTok{ }
\StringTok{  }\KeywordTok{select}\NormalTok{(}\OperatorTok{{-}}\NormalTok{classe\_assunto, }\OperatorTok{{-}}\NormalTok{txt\_decisao)}
\end{Highlighting}
\end{Shaded}

\begin{verbatim}
## # A tibble: 11,731 x 7
##    id_decisao n_processo   municipio  camara   data_decisao data_registro juiz  
##    <chr>      <chr>        <chr>      <chr>    <chr>        <chr>         <chr> 
##  1 11094999   0057003-20.~ Cosmópolis 3ª Câma~ 19/12/2017   19/12/2017    Luiz ~
##  2 11093733   0052762-03.~ São Paulo  3ª Câma~ 19/12/2017   19/12/2017    Luiz ~
##  3 11093677   0055169-79.~ Ribeirão ~ 3ª Câma~ 19/12/2017   19/12/2017    Luiz ~
##  4 11093270   9000580-82.~ Araçatuba  8ª Câma~ 14/12/2017   19/12/2017    Grass~
##  5 11093374   0052938-79.~ São Paulo  8ª Câma~ 14/12/2017   19/12/2017    Grass~
##  6 11093320   9000723-79.~ President~ 8ª Câma~ 14/12/2017   19/12/2017    Grass~
##  7 11091506   0003276-86.~ Bertioga   8ª Câma~ 14/12/2017   19/12/2017    Grass~
##  8 11093326   9000298-11.~ Taubaté    8ª Câma~ 14/12/2017   19/12/2017    Grass~
##  9 11092475   0004653-39.~ Aparecida  8ª Câma~ 14/12/2017   19/12/2017    Grass~
## 10 11093773   2221930-66.~ Jandira    3ª Câma~ 19/12/2017   19/12/2017    Luiz ~
## # ... with 11,721 more rows
\end{verbatim}
\end{frame}

\hypertarget{filter}{%
\section{filter}\label{filter}}

\begin{frame}[fragile]{\texttt{filter}}
\protect\hypertarget{filter-1}{}
\begin{itemize}
\tightlist
\item
  Use \texttt{,} ou \texttt{\&} para ``e'' e \texttt{\textbar{}} para
  ``ou''.
\item
  Condições separadas por vírgulas é o mesmo que separar por
  \texttt{\&}.
\end{itemize}
\end{frame}

\begin{frame}[fragile]{\texttt{filter} em ação}
\protect\hypertarget{filter-em-auxe7uxe3o}{}
\begin{Shaded}
\begin{Highlighting}[]
\NormalTok{decisoes }\OperatorTok{\%\textgreater{}\%}\StringTok{ }
\StringTok{  }\KeywordTok{select}\NormalTok{(n\_processo, id\_decisao, municipio, juiz) }\OperatorTok{\%\textgreater{}\%}\StringTok{ }
\StringTok{  }\KeywordTok{filter}\NormalTok{(municipio }\OperatorTok{==}\StringTok{ \textquotesingle{}São Paulo\textquotesingle{}}\NormalTok{)}
\end{Highlighting}
\end{Shaded}

\begin{verbatim}
## # A tibble: 2,446 x 4
##    n_processo                id_decisao municipio juiz                
##    <chr>                     <chr>      <chr>     <chr>               
##  1 0052762-03.2017.8.26.0000 11093733   São Paulo Luiz Antonio Cardoso
##  2 0052938-79.2017.8.26.0000 11093374   São Paulo Grassi Neto         
##  3 2214049-38.2017.8.26.0000 11093604   São Paulo Luiz Antonio Cardoso
##  4 2227499-48.2017.8.26.0000 11093642   São Paulo Luiz Antonio Cardoso
##  5 9002384-31.2017.8.26.0050 11093376   São Paulo Grassi Neto         
##  6 0021158-39.2015.8.26.0050 11091508   São Paulo Grassi Neto         
##  7 7005375-26.2015.8.26.0198 11091668   São Paulo Grassi Neto         
##  8 9002039-65.2017.8.26.0050 11094451   São Paulo Grassi Neto         
##  9 2203993-43.2017.8.26.0000 11094449   São Paulo Grassi Neto         
## 10 0099423-21.2016.8.26.0050 11091474   São Paulo Grassi Neto         
## # ... with 2,436 more rows
\end{verbatim}
\end{frame}

\begin{frame}[fragile]{Dica: usar \texttt{\%in\%}}
\protect\hypertarget{dica-usar-in}{}
\begin{Shaded}
\begin{Highlighting}[]
\KeywordTok{library}\NormalTok{(lubridate) }\CommentTok{\# para trabalhar com as datas}
\CommentTok{\#\textasciigrave{}day(dmy(data\_decisao))\textasciigrave{} pega o dia da decisão. }
\end{Highlighting}
\end{Shaded}

\begin{Shaded}
\begin{Highlighting}[]
\NormalTok{decisoes }\OperatorTok{\%\textgreater{}\%}\StringTok{ }
\StringTok{  }\KeywordTok{select}\NormalTok{(id\_decisao, municipio, data\_decisao, juiz) }\OperatorTok{\%\textgreater{}\%}\StringTok{ }
\StringTok{  }\CommentTok{\# municipio igual a campinas ou jaú, OU dia da decisão maior ou igual a 25}
\StringTok{  }\KeywordTok{filter}\NormalTok{(municipio }\OperatorTok{\%in\%}\StringTok{ }\KeywordTok{c}\NormalTok{(}\StringTok{\textquotesingle{}Campinas\textquotesingle{}}\NormalTok{, }\StringTok{\textquotesingle{}Jaú\textquotesingle{}}\NormalTok{) }\OperatorTok{|}\StringTok{ }\KeywordTok{day}\NormalTok{(}\KeywordTok{dmy}\NormalTok{(data\_decisao)) }\OperatorTok{\textgreater{}=}\StringTok{ }\DecValTok{25}\NormalTok{)}
\end{Highlighting}
\end{Shaded}

\begin{verbatim}
## # A tibble: 3,352 x 4
##    id_decisao municipio data_decisao juiz             
##    <chr>      <chr>     <chr>        <chr>            
##  1 11093272   Campinas  14/12/2017   Grassi Neto      
##  2 11093359   Campinas  07/12/2017   Grassi Neto      
##  3 11088333   Campinas  14/12/2017   Grassi Neto      
##  4 11093018   Jaú       28/11/2017   Ivan Sartori     
##  5 11089105   Jaú       14/12/2017   Ricardo Tucunduva
##  6 11089111   Campinas  14/12/2017   Ricardo Tucunduva
##  7 11091386   Santos    27/11/2017   Ivo de Almeida   
##  8 11091385   Araçatuba 27/11/2017   Ivo de Almeida   
##  9 11091384   São Paulo 27/11/2017   Ivo de Almeida   
## 10 11088180   Campinas  14/12/2017   Walter da Silva  
## # ... with 3,342 more rows
\end{verbatim}
\end{frame}

\begin{frame}[fragile]{Mais ação}
\protect\hypertarget{mais-auxe7uxe3o}{}
\begin{Shaded}
\begin{Highlighting}[]
\NormalTok{decisoes }\OperatorTok{\%\textgreater{}\%}\StringTok{ }
\StringTok{  }\KeywordTok{select}\NormalTok{(juiz) }\OperatorTok{\%\textgreater{}\%}\StringTok{ }
\StringTok{  }\CommentTok{\# filtra juízes que têm \textasciigrave{}Z\textasciigrave{} ou \textasciigrave{}z\textasciigrave{} no nome}
\StringTok{  }\KeywordTok{filter}\NormalTok{(}\KeywordTok{str\_detect}\NormalTok{(juiz, }\KeywordTok{regex}\NormalTok{(}\StringTok{"z"}\NormalTok{, }\DataTypeTok{ignore\_case =} \OtherTok{TRUE}\NormalTok{))) }\OperatorTok{\%\textgreater{}\%}\StringTok{ }
\StringTok{  }\CommentTok{\# conta e ordena os juizes em ordem decrescente}
\StringTok{  }\KeywordTok{count}\NormalTok{(juiz, }\DataTypeTok{sort =} \OtherTok{TRUE}\NormalTok{) }\OperatorTok{\%\textgreater{}\%}
\StringTok{  }\KeywordTok{head}\NormalTok{(}\DecValTok{5}\NormalTok{)}
\end{Highlighting}
\end{Shaded}

\begin{verbatim}
## # A tibble: 5 x 2
##   juiz                          n
##   <chr>                     <int>
## 1 Gilberto Ferreira da Cruz   237
## 2 Diniz Fernando              198
## 3 Sérgio Mazina Martins       173
## 4 Luiz Antonio Cardoso        163
## 5 Rachid Vaz de Almeida       150
\end{verbatim}
\end{frame}

\begin{frame}[fragile]{\textbf{Obs}}
\protect\hypertarget{obs}{}
A função \texttt{str\_detect()} retorna \texttt{TRUE} se um elemento do
vetor de textos é compatível com uma \emph{expressão regular}.
Estudaremos o pacote \texttt{stringr} e as funções \texttt{str\_*} em
outra aula.
\end{frame}

\begin{frame}[fragile]{}
\protect\hypertarget{section}{}
\begin{itemize}
\tightlist
\item
  filtre apenas casos em que \texttt{id\_decisao} não é \texttt{NA}
\end{itemize}

\begin{Shaded}
\begin{Highlighting}[]
\NormalTok{decisoes }\OperatorTok{\%\textgreater{}\%}\StringTok{ }
\StringTok{  }\KeywordTok{filter}\NormalTok{(}\KeywordTok{is.na}\NormalTok{(id\_decisao))}
\end{Highlighting}
\end{Shaded}

\begin{verbatim}
## # A tibble: 65 x 9
##    id_decisao n_processo classe_assunto municipio camara data_decisao
##    <chr>      <chr>      <chr>          <chr>     <chr>  <chr>       
##  1 <NA>       <NA>       <NA>           <NA>      <NA>   <NA>        
##  2 <NA>       <NA>       <NA>           <NA>      <NA>   <NA>        
##  3 <NA>       <NA>       <NA>           <NA>      <NA>   <NA>        
##  4 <NA>       <NA>       <NA>           <NA>      <NA>   <NA>        
##  5 <NA>       <NA>       <NA>           <NA>      <NA>   <NA>        
##  6 <NA>       <NA>       <NA>           <NA>      <NA>   <NA>        
##  7 <NA>       <NA>       <NA>           <NA>      <NA>   <NA>        
##  8 <NA>       <NA>       <NA>           <NA>      <NA>   <NA>        
##  9 <NA>       <NA>       <NA>           <NA>      <NA>   <NA>        
## 10 <NA>       <NA>       <NA>           <NA>      <NA>   <NA>        
## # ... with 55 more rows, and 3 more variables: data_registro <chr>, juiz <chr>,
## #   txt_decisao <chr>
\end{verbatim}
\end{frame}

\begin{frame}[fragile]{}
\protect\hypertarget{section-1}{}
\begin{itemize}
\item
  filtre todas as decisões de 2018.

  -- Dica: função \texttt{lubridate::year()}
\end{itemize}

\begin{Shaded}
\begin{Highlighting}[]
\NormalTok{decisoes }\OperatorTok{\%\textgreater{}\%}\StringTok{ }
\StringTok{  }\KeywordTok{filter}\NormalTok{(}\KeywordTok{year}\NormalTok{(}\KeywordTok{dmy}\NormalTok{(data\_decisao)) }\OperatorTok{==}\StringTok{ }\DecValTok{2018}\NormalTok{)}
\end{Highlighting}
\end{Shaded}

\begin{verbatim}
## # A tibble: 314 x 9
##    id_decisao n_processo classe_assunto municipio camara data_decisao
##    <chr>      <chr>      <chr>          <chr>     <chr>  <chr>       
##  1 11107242   0009617-6~ Apelação / Ro~ São Paulo 2ª Câ~ 15/01/2018  
##  2 11107425   2227593-9~ Habeas Corpus~ Iepê      2ª Câ~ 15/01/2018  
##  3 11107492   0076977-2~ Embargos de D~ São Paulo 2ª Câ~ 15/01/2018  
##  4 11107361   0012191-3~ Agravo de Exe~ Campinas  2ª Câ~ 15/01/2018  
##  5 11107383   2218460-2~ Habeas Corpus~ Sorocaba  2ª Câ~ 15/01/2018  
##  6 11107331   0006928-6~ Agravo de Exe~ Sorocaba  2ª Câ~ 15/01/2018  
##  7 11107651   0000297-5~ Apelação / Tr~ Junqueir~ 2ª Câ~ 15/01/2018  
##  8 11107485   2225548-1~ Habeas Corpus~ Nazaré P~ 2ª Câ~ 15/01/2018  
##  9 11107335   0006934-7~ Agravo de Exe~ Sorocaba  2ª Câ~ 15/01/2018  
## 10 11107340   0006682-6~ Agravo de Exe~ Sorocaba  2ª Câ~ 15/01/2018  
## # ... with 304 more rows, and 3 more variables: data_registro <chr>,
## #   juiz <chr>, txt_decisao <chr>
\end{verbatim}
\end{frame}

\hypertarget{mutate}{%
\section{mutate}\label{mutate}}

\begin{frame}[fragile]{\texttt{mutate}}
\protect\hypertarget{mutate-1}{}
\begin{itemize}
\item
  Aceita várias novas colunas iterativamente.
\item
  Novas variáveis devem ter o mesmo \texttt{length} que o \texttt{nrow}
  do bd original ou \texttt{1}.
\end{itemize}
\end{frame}

\begin{frame}[fragile]{\texttt{mutate} em ação}
\protect\hypertarget{mutate-em-auxe7uxe3o}{}
\begin{Shaded}
\begin{Highlighting}[]
\NormalTok{decisoes }\OperatorTok{\%\textgreater{}\%}\StringTok{ }
\StringTok{  }\KeywordTok{select}\NormalTok{(n\_processo, data\_decisao, data\_registro) }\OperatorTok{\%\textgreater{}\%}\StringTok{ }
\StringTok{  }\KeywordTok{mutate}\NormalTok{(}\DataTypeTok{tempo =} \KeywordTok{dmy}\NormalTok{(data\_registro) }\OperatorTok{{-}}\StringTok{ }\KeywordTok{dmy}\NormalTok{(data\_decisao))}
\end{Highlighting}
\end{Shaded}

\begin{verbatim}
## # A tibble: 11,731 x 4
##    n_processo                data_decisao data_registro tempo 
##    <chr>                     <chr>        <chr>         <drtn>
##  1 0057003-20.2017.8.26.0000 19/12/2017   19/12/2017    0 days
##  2 0052762-03.2017.8.26.0000 19/12/2017   19/12/2017    0 days
##  3 0055169-79.2017.8.26.0000 19/12/2017   19/12/2017    0 days
##  4 9000580-82.2017.8.26.0032 14/12/2017   19/12/2017    5 days
##  5 0052938-79.2017.8.26.0000 14/12/2017   19/12/2017    5 days
##  6 9000723-79.2017.8.26.0482 14/12/2017   19/12/2017    5 days
##  7 0003276-86.2015.8.26.0075 14/12/2017   19/12/2017    5 days
##  8 9000298-11.2017.8.26.0625 14/12/2017   19/12/2017    5 days
##  9 0004653-39.2015.8.26.0028 14/12/2017   19/12/2017    5 days
## 10 2221930-66.2017.8.26.0000 19/12/2017   19/12/2017    0 days
## # ... with 11,721 more rows
\end{verbatim}
\end{frame}

\begin{frame}[fragile]{}
\protect\hypertarget{section-2}{}
\begin{itemize}
\tightlist
\item
  Crie uma coluna binária \texttt{drogas} que vale \texttt{TRUE} se no
  texto da decisão algo é falado de drogas e \texttt{FALSE} caso
  contrário. -- Dica: \texttt{str\_detect}
\end{itemize}

Obs.: Considere tanto a palavra `droga' como seus sinônimos, ou algum
exemplo de droga e retire os casos em que \texttt{txt\_decisao} é vazio

\begin{Shaded}
\begin{Highlighting}[]
\NormalTok{decisoes }\OperatorTok{\%\textgreater{}\%}\StringTok{ }
\StringTok{  }\KeywordTok{filter}\NormalTok{(}\OperatorTok{!}\KeywordTok{is.na}\NormalTok{(txt\_decisao)) }\OperatorTok{\%\textgreater{}\%}\StringTok{ }
\StringTok{  }\KeywordTok{mutate}\NormalTok{(}\DataTypeTok{txt\_decisao =} \KeywordTok{tolower}\NormalTok{(txt\_decisao),}
         \DataTypeTok{droga =} \KeywordTok{str\_detect}\NormalTok{(txt\_decisao,}
    \StringTok{"droga|entorpecente|psicotr[óo]pico|maconha|haxixe|coca[íi]na"}\NormalTok{)) }\OperatorTok{\%\textgreater{}\%}
\StringTok{  }\NormalTok{dplyr}\OperatorTok{::}\KeywordTok{select}\NormalTok{(n\_processo,droga) }
\end{Highlighting}
\end{Shaded}

\begin{verbatim}
## # A tibble: 6,933 x 2
##    n_processo                droga
##    <chr>                     <lgl>
##  1 9000580-82.2017.8.26.0032 FALSE
##  2 0052938-79.2017.8.26.0000 FALSE
##  3 9000723-79.2017.8.26.0482 FALSE
##  4 0003276-86.2015.8.26.0075 TRUE 
##  5 9000298-11.2017.8.26.0625 TRUE 
##  6 0004653-39.2015.8.26.0028 TRUE 
##  7 9000788-34.2017.8.26.0269 FALSE
##  8 9000673-53.2017.8.26.0482 FALSE
##  9 7005620-87.2017.8.26.0482 FALSE
## 10 7001988-32.2017.8.26.0198 FALSE
## # ... with 6,923 more rows
\end{verbatim}
\end{frame}

\hypertarget{summarise}{%
\section{summarise}\label{summarise}}

\begin{frame}[fragile]{\texttt{summarise}}
\protect\hypertarget{summarise-1}{}
\begin{itemize}
\tightlist
\item
  Retorna um vetor de tamanho \texttt{1} a partir de uma operação com as
  variáveis (aplicação de uma função).
\item
  Geralmente é utilizado em conjunto com \texttt{group\_by()}.
\item
  Algumas funções importantes: \texttt{n()}, \texttt{n\_distinct()}.
\end{itemize}
\end{frame}

\begin{frame}[fragile]{Em ação}
\protect\hypertarget{em-auxe7uxe3o-3}{}
\begin{Shaded}
\begin{Highlighting}[]
\NormalTok{decisoes }\OperatorTok{\%\textgreater{}\%}\StringTok{ }
\StringTok{  }\KeywordTok{select}\NormalTok{(n\_processo, municipio, data\_decisao) }\OperatorTok{\%\textgreater{}\%}
\StringTok{  }\CommentTok{\#        pega ano da decisão}
\StringTok{  }\KeywordTok{mutate}\NormalTok{(}\DataTypeTok{ano\_julgamento =} \KeywordTok{year}\NormalTok{(}\KeywordTok{dmy}\NormalTok{(data\_decisao)),}
         \CommentTok{\# pega o ano do processo 0057003{-}20.2017.8.26.0000" {-}\textgreater{} "2017"}
         \DataTypeTok{ano\_proc =} \KeywordTok{str\_sub}\NormalTok{(n\_processo, }\DecValTok{12}\NormalTok{, }\DecValTok{15}\NormalTok{),}
         \CommentTok{\# transforma o ano em inteiro}
         \DataTypeTok{ano\_proc =} \KeywordTok{as.numeric}\NormalTok{(ano\_proc),}
         \CommentTok{\# calcula o tempo em anos}
         \DataTypeTok{tempo\_anos =}\NormalTok{ ano\_julgamento }\OperatorTok{{-}}\StringTok{ }\NormalTok{ano\_proc) }\OperatorTok{\%\textgreater{}\%}\StringTok{ }
\StringTok{  }\KeywordTok{group\_by}\NormalTok{(municipio) }\OperatorTok{\%\textgreater{}\%}\StringTok{ }
\StringTok{  }\KeywordTok{summarise}\NormalTok{(}\DataTypeTok{n =} \KeywordTok{n}\NormalTok{(),}
            \DataTypeTok{media\_anos =} \KeywordTok{mean}\NormalTok{(tempo\_anos),}
            \DataTypeTok{min\_anos =} \KeywordTok{min}\NormalTok{(tempo\_anos),}
            \DataTypeTok{max\_anos =} \KeywordTok{max}\NormalTok{(tempo\_anos)) }
\end{Highlighting}
\end{Shaded}
\end{frame}

\begin{frame}[fragile]{Resultado}
\protect\hypertarget{resultado}{}
\begin{verbatim}
## `summarise()` ungrouping output (override with `.groups` argument)
\end{verbatim}

\begin{verbatim}
## # A tibble: 315 x 5
##    municipio               n media_anos min_anos max_anos
##    <chr>               <int>      <dbl>    <dbl>    <dbl>
##  1 Adamantina             17      0.765        0        5
##  2 Aguaí                  19      1.16         0        2
##  3 Águas de Lindóia        5      1.4          0        4
##  4 Agudos                  8      3.25         0        6
##  5 Altinópolis             7      0.857        0        2
##  6 Americana              56      1.41         0        9
##  7 Américo Brasiliense     9      1.56         0        3
##  8 Amparo                  9      2.11         0        7
##  9 Andradina              41      0.707        0        5
## 10 Angatuba                4      0.5          0        2
## # ... with 305 more rows
\end{verbatim}
\end{frame}

\begin{frame}[fragile]{usando \texttt{count()}}
\protect\hypertarget{usando-count}{}
A função \texttt{count()}, simplifica um
\texttt{group\_by\ \%\textgreater{}\%\ summarise\ \%\textgreater{}\%\ ungroup}:

\begin{Shaded}
\begin{Highlighting}[]
\NormalTok{decisoes }\OperatorTok{\%\textgreater{}\%}\StringTok{ }
\StringTok{  }\KeywordTok{count}\NormalTok{(juiz, }\DataTypeTok{sort =} \OtherTok{TRUE}\NormalTok{) }\OperatorTok{\%\textgreater{}\%}\StringTok{ }
\StringTok{  }\KeywordTok{mutate}\NormalTok{(}\DataTypeTok{prop =}\NormalTok{ n }\OperatorTok{/}\StringTok{ }\KeywordTok{sum}\NormalTok{(n), }
         \DataTypeTok{prop =}\NormalTok{ scales}\OperatorTok{::}\KeywordTok{percent}\NormalTok{(prop))}
\end{Highlighting}
\end{Shaded}

\begin{verbatim}
## # A tibble: 100 x 3
##    juiz                          n prop   
##    <chr>                     <int> <chr>  
##  1 Gilberto Ferreira da Cruz   237 2.0203%
##  2 Francisco Orlando           226 1.9265%
##  3 Diniz Fernando              198 1.6878%
##  4 Walter da Silva             183 1.5600%
##  5 De Paula Santos             182 1.5514%
##  6 Machado de Andrade          182 1.5514%
##  7 Newton Neves                180 1.5344%
##  8 Leme Garcia                 179 1.5259%
##  9 Grassi Neto                 177 1.5088%
## 10 Figueiredo Gonçalves        176 1.5003%
## # ... with 90 more rows
\end{verbatim}
\end{frame}

\begin{frame}[fragile]{+ fácil ainda}
\protect\hypertarget{fuxe1cil-ainda}{}
mas sem formato \%

\begin{Shaded}
\begin{Highlighting}[]
\NormalTok{decisoes }\OperatorTok{\%\textgreater{}\%}\StringTok{ }
\StringTok{  }\KeywordTok{count}\NormalTok{(juiz, }\DataTypeTok{sort =} \OtherTok{TRUE}\NormalTok{) }\OperatorTok{\%\textgreater{}\%}\StringTok{ }
\StringTok{  }\KeywordTok{mutate}\NormalTok{(}\DataTypeTok{prop =} \KeywordTok{prop.table}\NormalTok{(n))}
\end{Highlighting}
\end{Shaded}

\begin{verbatim}
## # A tibble: 100 x 3
##    juiz                          n   prop
##    <chr>                     <int>  <dbl>
##  1 Gilberto Ferreira da Cruz   237 0.0202
##  2 Francisco Orlando           226 0.0193
##  3 Diniz Fernando              198 0.0169
##  4 Walter da Silva             183 0.0156
##  5 De Paula Santos             182 0.0155
##  6 Machado de Andrade          182 0.0155
##  7 Newton Neves                180 0.0153
##  8 Leme Garcia                 179 0.0153
##  9 Grassi Neto                 177 0.0151
## 10 Figueiredo Gonçalves        176 0.0150
## # ... with 90 more rows
\end{verbatim}
\end{frame}

\hypertarget{arrange}{%
\section{arrange}\label{arrange}}

\begin{frame}[fragile]{\texttt{arrange}}
\protect\hypertarget{arrange-1}{}
\begin{itemize}
\item
  Simplesmente ordena de acordo com as opções.
\item
  Utilizar \texttt{desc()} para ordem decrescente ou o sinal de menos
  (\texttt{-}).
\end{itemize}
\end{frame}

\begin{frame}[fragile]{}
\protect\hypertarget{section-3}{}
\begin{itemize}
\tightlist
\item
  Quem são os cinco relatores mais prolixos?
\end{itemize}

-- Dica: use \texttt{str\_length()} -- Lembre-se da função
\texttt{head()}
\end{frame}

\begin{frame}[fragile]{}
\protect\hypertarget{section-4}{}
\begin{Shaded}
\begin{Highlighting}[]
\NormalTok{decisoes }\OperatorTok{\%\textgreater{}\%}\StringTok{ }
\StringTok{  }\KeywordTok{filter}\NormalTok{(}\OperatorTok{!}\KeywordTok{is.na}\NormalTok{(txt\_decisao)) }\OperatorTok{\%\textgreater{}\%}\StringTok{ }
\StringTok{  }\KeywordTok{mutate}\NormalTok{(}\DataTypeTok{tamanho =} \KeywordTok{str\_length}\NormalTok{(txt\_decisao)) }\OperatorTok{\%\textgreater{}\%}\StringTok{ }
\StringTok{  }\KeywordTok{group\_by}\NormalTok{(juiz) }\OperatorTok{\%\textgreater{}\%}\StringTok{ }
\StringTok{  }\KeywordTok{summarise}\NormalTok{(}\DataTypeTok{n =} \KeywordTok{n}\NormalTok{(), }
            \DataTypeTok{tamanho\_mediana =} \KeywordTok{median}\NormalTok{(tamanho)) }\OperatorTok{\%\textgreater{}\%}\StringTok{ }
\StringTok{  }\KeywordTok{filter}\NormalTok{(n }\OperatorTok{\textgreater{}=}\StringTok{ }\DecValTok{10}\NormalTok{) }\OperatorTok{\%\textgreater{}\%}\StringTok{ }
\StringTok{  }\KeywordTok{arrange}\NormalTok{(}\KeywordTok{desc}\NormalTok{(tamanho\_mediana)) }\OperatorTok{\%\textgreater{}\%}
\StringTok{  }\KeywordTok{head}\NormalTok{()}
\end{Highlighting}
\end{Shaded}

\begin{verbatim}
## `summarise()` ungrouping output (override with `.groups` argument)
\end{verbatim}

\begin{verbatim}
## # A tibble: 6 x 3
##   juiz                                n tamanho_mediana
##   <chr>                           <int>           <dbl>
## 1 Airton Vieira                     154           3146.
## 2 Ely Amioka                         81           1847 
## 3 Grassi Neto                       141           1675 
## 4 Alcides Malossi Junior             95           1541 
## 5 Cesar Augusto Andrade de Castro    77           1341 
## 6 Juvenal Duarte                    101           1320
\end{verbatim}
\end{frame}

\hypertarget{tydyr}{%
\section{tydyr}\label{tydyr}}

\begin{frame}[fragile]{Alterando o formato de dados}
\protect\hypertarget{alterando-o-formato-de-dados}{}
Até agora, estudamos os principais ferramentas de transformação de dados
do \texttt{dplyr}. Agora vamos aumentar nossa caixa-de-ferramentas com
\texttt{tidyr}

\begin{itemize}
\tightlist
\item
  Carregando uma nova base de dados, que completa a de decisões.
\end{itemize}

\begin{Shaded}
\begin{Highlighting}[]
\NormalTok{processos \textless{}{-}}\StringTok{ }\KeywordTok{read\_rds}\NormalTok{(}\StringTok{"dados/processos\_nested.rds"}\NormalTok{)}
\end{Highlighting}
\end{Shaded}
\end{frame}

\begin{frame}{Fomato tidy}
\protect\hypertarget{fomato-tidy}{}
\begin{itemize}
\tightlist
\item
  Hadley Wickham \url{http://r4ds.had.co.nz/tidy-data.html}
\end{itemize}
\end{frame}

\begin{frame}[fragile]{Funções do pacote}
\protect\hypertarget{funuxe7uxf5es-do-pacote}{}
\begin{itemize}
\item
  Enquanto o \texttt{dplyr} faz recortes na base (com \texttt{filter()}e
  \texttt{select()}) e adições simples (\texttt{mutate()},
  \texttt{summarise()}), o \texttt{tidyr} mexe no \textbf{formato} da
  tabela (\texttt{gather()}, \texttt{spread()}) e faz modificações menos
  triviais.
\item
  As funções do \texttt{tidyr} geralmente vêm em pares com seus
  inversos:

  \begin{itemize}
  \tightlist
  \item
    \texttt{gather()} e \texttt{spread()}, -\textgreater{} substituídas
    por \texttt{pivot\_longer} e \texttt{pivot\_wider}
  \item
    \texttt{nest()} e \texttt{unnest()},
  \item
    \texttt{separate()} e \texttt{unite()}
  \end{itemize}
\end{itemize}
\end{frame}

\begin{frame}{Onde estamos}
\protect\hypertarget{onde-estamos}{}
\url{http://r4ds.had.co.nz/wrangle-intro.html}
\end{frame}

\begin{frame}[fragile]{\texttt{gather()}}
\protect\hypertarget{gather}{}
\begin{itemize}
\item
  \texttt{gather()} empilha o banco de dados
\item
  \texttt{pivot\_longer} empilha de um jeito ainda mais fácil
\end{itemize}

\begin{Shaded}
\begin{Highlighting}[]
\NormalTok{decisoes }\OperatorTok{\%\textgreater{}\%}\StringTok{ }
\StringTok{  }\KeywordTok{filter}\NormalTok{(}\OperatorTok{!}\KeywordTok{is.na}\NormalTok{(id\_decisao)) }\OperatorTok{\%\textgreater{}\%}\StringTok{ }
\StringTok{  }\KeywordTok{select}\NormalTok{(id\_decisao}\OperatorTok{:}\NormalTok{data\_registro) }\OperatorTok{\%\textgreater{}\%}\StringTok{ }
\StringTok{  }\CommentTok{\# 1. nome da coluna que vai guardar os nomes de colunas empilhadas}
\StringTok{  }\CommentTok{\# 2. nome da coluna que vai guardar os valores das colunas}
\StringTok{  }\CommentTok{\# 3. seleção das colunas a serem empilhadas}
\StringTok{  }\KeywordTok{gather}\NormalTok{(}\DataTypeTok{key=}\StringTok{"variavel"}\NormalTok{, }\DataTypeTok{value=}\StringTok{"valor"}\NormalTok{, }\OperatorTok{{-}}\NormalTok{id\_decisao) }\OperatorTok{\%\textgreater{}\%}\StringTok{ }
\StringTok{  }\KeywordTok{arrange}\NormalTok{(id\_decisao)}
\end{Highlighting}
\end{Shaded}

\begin{verbatim}
## # A tibble: 69,996 x 3
##    id_decisao variavel       valor                                        
##    <chr>      <chr>          <chr>                                        
##  1 11026431   n_processo     0000009-51.2015.8.26.0546                    
##  2 11026431   classe_assunto Apelação / Tráfico de Drogas e Condutas Afins
##  3 11026431   municipio      Itapira                                      
##  4 11026431   camara         5ª Câmara de Direito Criminal                
##  5 11026431   data_decisao   30/11/2017                                   
##  6 11026431   data_registro  01/12/2017                                   
##  7 11026432   n_processo     0002267-69.2013.8.26.0654                    
##  8 11026432   classe_assunto Apelação / Furto Qualificado                 
##  9 11026432   municipio      Vargem Grande Paulista                       
## 10 11026432   camara         5ª Câmara de Direito Criminal                
## # ... with 69,986 more rows
\end{verbatim}
\end{frame}

\begin{frame}[fragile]{\texttt{pivot\_longer}}
\protect\hypertarget{pivot_longer}{}
Base \texttt{relig\_income} do \texttt{tidyr} 3 variáveis:

\begin{itemize}
\tightlist
\item
  religion, nas linhas
\item
  income, nas colunas e
\item
  count, nas células
\end{itemize}

\begin{Shaded}
\begin{Highlighting}[]
\NormalTok{tidyr}\OperatorTok{::}\NormalTok{relig\_income}
\end{Highlighting}
\end{Shaded}

\begin{verbatim}
## # A tibble: 18 x 11
##    religion `<$10k` `$10-20k` `$20-30k` `$30-40k` `$40-50k` `$50-75k` `$75-100k`
##    <chr>      <dbl>     <dbl>     <dbl>     <dbl>     <dbl>     <dbl>      <dbl>
##  1 Agnostic      27        34        60        81        76       137        122
##  2 Atheist       12        27        37        52        35        70         73
##  3 Buddhist      27        21        30        34        33        58         62
##  4 Catholic     418       617       732       670       638      1116        949
##  5 Don’t k~      15        14        15        11        10        35         21
##  6 Evangel~     575       869      1064       982       881      1486        949
##  7 Hindu          1         9         7         9        11        34         47
##  8 Histori~     228       244       236       238       197       223        131
##  9 Jehovah~      20        27        24        24        21        30         15
## 10 Jewish        19        19        25        25        30        95         69
## 11 Mainlin~     289       495       619       655       651      1107        939
## 12 Mormon        29        40        48        51        56       112         85
## 13 Muslim         6         7         9        10         9        23         16
## 14 Orthodox      13        17        23        32        32        47         38
## 15 Other C~       9         7        11        13        13        14         18
## 16 Other F~      20        33        40        46        49        63         46
## 17 Other W~       5         2         3         4         2         7          3
## 18 Unaffil~     217       299       374       365       341       528        407
## # ... with 3 more variables: `$100-150k` <dbl>, `>150k` <dbl>, `Don't
## #   know/refused` <dbl>
\end{verbatim}
\end{frame}

\begin{frame}[fragile]{}
\protect\hypertarget{section-5}{}
\begin{Shaded}
\begin{Highlighting}[]
\NormalTok{tidyr}\OperatorTok{::}\NormalTok{relig\_income }\OperatorTok{\%\textgreater{}\%}
\StringTok{  }\KeywordTok{pivot\_longer}\NormalTok{(}\OperatorTok{!}\NormalTok{religion, }
               \DataTypeTok{names\_to =} \StringTok{"income"}\NormalTok{, }\CommentTok{\# diz a varável onde entrarão os nomes}
               \DataTypeTok{values\_to =} \StringTok{"count"} \CommentTok{\# diz a variável onde entrarão os valores}
\NormalTok{               )}
\end{Highlighting}
\end{Shaded}

\begin{verbatim}
## # A tibble: 180 x 3
##    religion income             count
##    <chr>    <chr>              <dbl>
##  1 Agnostic <$10k                 27
##  2 Agnostic $10-20k               34
##  3 Agnostic $20-30k               60
##  4 Agnostic $30-40k               81
##  5 Agnostic $40-50k               76
##  6 Agnostic $50-75k              137
##  7 Agnostic $75-100k             122
##  8 Agnostic $100-150k            109
##  9 Agnostic >150k                 84
## 10 Agnostic Don't know/refused    96
## # ... with 170 more rows
\end{verbatim}
\end{frame}

\begin{frame}[fragile]{\texttt{spread()}}
\protect\hypertarget{spread}{}
\begin{itemize}
\item
  \texttt{spread()} espalha uma variável nas colunas e preenche com
  outra variável
\item
  Função inversa de \texttt{gather}
\item
  Bem mais fácil com \texttt{pivot\_wider}
\end{itemize}

\begin{Shaded}
\begin{Highlighting}[]
\NormalTok{decisoes }\OperatorTok{\%\textgreater{}\%}\StringTok{ }
\StringTok{  }\KeywordTok{filter}\NormalTok{(}\OperatorTok{!}\KeywordTok{is.na}\NormalTok{(id\_decisao)) }\OperatorTok{\%\textgreater{}\%}\StringTok{ }
\StringTok{  }\KeywordTok{select}\NormalTok{(id\_decisao}\OperatorTok{:}\NormalTok{data\_registro) }\OperatorTok{\%\textgreater{}\%}\StringTok{ }
\StringTok{  }\KeywordTok{gather}\NormalTok{(key, value, }\OperatorTok{{-}}\NormalTok{id\_decisao) }\OperatorTok{\%\textgreater{}\%}\StringTok{ }
\StringTok{  }\CommentTok{\# 1. coluna a ser espalhada}
\StringTok{  }\CommentTok{\# 2. valores da coluna}
\StringTok{  }\KeywordTok{spread}\NormalTok{(key, value)}
\end{Highlighting}
\end{Shaded}

\begin{verbatim}
## # A tibble: 11,666 x 7
##    id_decisao camara classe_assunto data_decisao data_registro municipio
##    <chr>      <chr>  <chr>          <chr>        <chr>         <chr>    
##  1 11026431   5ª Câ~ Apelação / Tr~ 30/11/2017   01/12/2017    Itapira  
##  2 11026432   5ª Câ~ Apelação / Fu~ 30/11/2017   01/12/2017    Vargem G~
##  3 11026433   5ª Câ~ Apelação / Ro~ 30/11/2017   01/12/2017    Sertãozi~
##  4 11026434   12ª C~ Agravo de Exe~ 18/10/2017   01/12/2017    Ribeirão~
##  5 11026435   15ª C~ Apelação / Tr~ 30/11/2017   01/12/2017    São Paulo
##  6 11026442   5ª Câ~ Apelação / Es~ 30/11/2017   01/12/2017    Jales    
##  7 11026445   13ª C~ Apelação / Us~ 30/11/2017   01/12/2017    Santa Fé~
##  8 11026453   5ª Câ~ Apelação / Cr~ 30/11/2017   01/12/2017    Franca   
##  9 11026455   12ª C~ Apelação / Ro~ 05/07/2017   01/12/2017    Franca   
## 10 11026456   12ª C~ Apelação / Ro~ 03/05/2017   01/12/2017    São José~
## # ... with 11,656 more rows, and 1 more variable: n_processo <chr>
\end{verbatim}
\end{frame}

\begin{frame}[fragile]{\texttt{pivot\_wider}}
\protect\hypertarget{pivot_wider}{}
\begin{Shaded}
\begin{Highlighting}[]
\NormalTok{decisoes }\OperatorTok{\%\textgreater{}\%}\StringTok{ }
\StringTok{  }\KeywordTok{filter}\NormalTok{(}\OperatorTok{!}\KeywordTok{is.na}\NormalTok{(id\_decisao)) }\OperatorTok{\%\textgreater{}\%}\StringTok{ }
\StringTok{  }\KeywordTok{select}\NormalTok{(id\_decisao}\OperatorTok{:}\NormalTok{data\_registro) }\OperatorTok{\%\textgreater{}\%}\StringTok{ }
\StringTok{  }\KeywordTok{pivot\_longer}\NormalTok{(}\OperatorTok{!}\NormalTok{id\_decisao, }
               \DataTypeTok{names\_to =} \StringTok{"tipo"}\NormalTok{, }\CommentTok{\# diz a varável onde entrarão os nomes}
               \DataTypeTok{values\_to =} \StringTok{"info"} \CommentTok{\# diz a variável onde entrarão os valores}
\NormalTok{               ) }\OperatorTok{\%\textgreater{}\%}\StringTok{ }
\StringTok{  }\CommentTok{\# 1. coluna a ser espalhada}
\StringTok{  }\CommentTok{\# 2. valores da coluna}
\StringTok{  }\KeywordTok{pivot\_wider}\NormalTok{(}\DataTypeTok{names\_from =} \StringTok{"tipo"}\NormalTok{,}
              \DataTypeTok{values\_from =} \StringTok{"info"}\NormalTok{)}
\end{Highlighting}
\end{Shaded}

\begin{verbatim}
## # A tibble: 11,666 x 7
##    id_decisao n_processo classe_assunto municipio camara data_decisao
##    <chr>      <chr>      <chr>          <chr>     <chr>  <chr>       
##  1 11094999   0057003-2~ Habeas Corpus~ Cosmópol~ 3ª Câ~ 19/12/2017  
##  2 11093733   0052762-0~ Habeas Corpus~ São Paulo 3ª Câ~ 19/12/2017  
##  3 11093677   0055169-7~ Habeas Corpus~ Ribeirão~ 3ª Câ~ 19/12/2017  
##  4 11093270   9000580-8~ Agravo de Exe~ Araçatuba 8ª Câ~ 14/12/2017  
##  5 11093374   0052938-7~ Mandado de Se~ São Paulo 8ª Câ~ 14/12/2017  
##  6 11093320   9000723-7~ Agravo de Exe~ Presiden~ 8ª Câ~ 14/12/2017  
##  7 11091506   0003276-8~ Apelação / Tr~ Bertioga  8ª Câ~ 14/12/2017  
##  8 11093326   9000298-1~ Agravo de Exe~ Taubaté   8ª Câ~ 14/12/2017  
##  9 11092475   0004653-3~ Apelação / Tr~ Aparecida 8ª Câ~ 14/12/2017  
## 10 11093773   2221930-6~ Habeas Corpus~ Jandira   3ª Câ~ 19/12/2017  
## # ... with 11,656 more rows, and 1 more variable: data_registro <chr>
\end{verbatim}
\end{frame}

\begin{frame}[fragile]{}
\protect\hypertarget{section-6}{}
\begin{itemize}
\tightlist
\item
  Qual juiz julga a maior proporção de processos que tratam de drogas
\end{itemize}

-- Dica: construa um \texttt{data.frame} contendo as colunas juiz,
n\_processos\_drogas, n\_processos\_n\_drogas e total\_processos,
remodelando os dados para haver um juiz por linha e utilizando
\texttt{spread()}
\end{frame}

\begin{frame}[fragile]{Resolução}
\protect\hypertarget{resoluuxe7uxe3o}{}
\begin{verbatim}
## `summarise()` regrouping output by 'juiz' (override with `.groups` argument)
\end{verbatim}

\begin{verbatim}
## # A tibble: 65 x 5
## # Groups:   juiz [65]
##    juiz                   droga n_droga total proporcao
##    <chr>                  <dbl>   <dbl> <dbl>     <dbl>
##  1 Airton Vieira             23     131   154     0.149
##  2 Alcides Malossi Junior    23      72    95     0.242
##  3 Alexandre Almeida         41     122   163     0.252
##  4 Amaro Thomé               36      96   132     0.273
##  5 Andrade Sampaio           35      79   114     0.307
##  6 Angélica de Almeida        2       6     8     0.25 
##  7 Antonio Tadeu Ottoni       0       1     1     0    
##  8 Bandeira Lins              0       2     2     0    
##  9 Camargo Aranha Filho      32     109   141     0.227
## 10 Camilo Léllis             32     133   165     0.194
## # ... with 55 more rows
\end{verbatim}
\end{frame}

\begin{frame}[fragile]{Exercício}
\protect\hypertarget{exercuxedcio}{}
\begin{itemize}
\item
  Qual quantidade mensal de decisões por juiz?
\item
  Dica: use \texttt{data\_decisao} \texttt{dmy()} e \texttt{month()}
\end{itemize}
\end{frame}

\begin{frame}[fragile]{Resolução}
\protect\hypertarget{resoluuxe7uxe3o-1}{}
\begin{Shaded}
\begin{Highlighting}[]
\NormalTok{decisoes }\OperatorTok{\%\textgreater{}\%}\StringTok{ }
\StringTok{  }\KeywordTok{filter}\NormalTok{(}\OperatorTok{!}\KeywordTok{is.na}\NormalTok{(txt\_decisao)) }\OperatorTok{\%\textgreater{}\%}
\StringTok{  }\KeywordTok{mutate}\NormalTok{(}\DataTypeTok{txt\_decisao =} \KeywordTok{tolower}\NormalTok{(txt\_decisao),}
         \DataTypeTok{droga =} \KeywordTok{str\_detect}\NormalTok{(txt\_decisao,}
    \StringTok{"droga|entorpecente|psicotr[óo]pico|maconha|haxixe|coca[íi]na"}\NormalTok{),}
    \DataTypeTok{droga=}\KeywordTok{case\_when}\NormalTok{(}
\NormalTok{      droga}\OperatorTok{==}\OtherTok{TRUE} \OperatorTok{\textasciitilde{}}\StringTok{ "droga"}\NormalTok{,}
\NormalTok{      droga}\OperatorTok{==}\OtherTok{FALSE} \OperatorTok{\textasciitilde{}}\StringTok{ "n\_droga"}
\NormalTok{    )) }\OperatorTok{\%\textgreater{}\%}
\StringTok{  }\KeywordTok{group\_by}\NormalTok{(juiz,droga) }\OperatorTok{\%\textgreater{}\%}
\StringTok{  }\KeywordTok{summarise}\NormalTok{(}\DataTypeTok{n=}\KeywordTok{n}\NormalTok{()) }\OperatorTok{\%\textgreater{}\%}
\StringTok{  }\KeywordTok{spread}\NormalTok{(droga,n,}\DataTypeTok{fill =} \DecValTok{0}\NormalTok{) }\OperatorTok{\%\textgreater{}\%}
\StringTok{  }\KeywordTok{mutate}\NormalTok{(}\DataTypeTok{total=}\NormalTok{droga}\OperatorTok{+}\NormalTok{n\_droga,}
         \DataTypeTok{proporcao=}\NormalTok{droga}\OperatorTok{/}\NormalTok{total)}
\end{Highlighting}
\end{Shaded}
\end{frame}

\begin{frame}[fragile]{Resultado}
\protect\hypertarget{resultado-1}{}
\begin{verbatim}
## `summarise()` regrouping output by 'juiz' (override with `.groups` argument)
\end{verbatim}

\begin{verbatim}
## # A tibble: 65 x 5
## # Groups:   juiz [65]
##    juiz                   droga n_droga total proporcao
##    <chr>                  <dbl>   <dbl> <dbl>     <dbl>
##  1 Airton Vieira             23     131   154     0.149
##  2 Alcides Malossi Junior    23      72    95     0.242
##  3 Alexandre Almeida         41     122   163     0.252
##  4 Amaro Thomé               36      96   132     0.273
##  5 Andrade Sampaio           35      79   114     0.307
##  6 Angélica de Almeida        2       6     8     0.25 
##  7 Antonio Tadeu Ottoni       0       1     1     0    
##  8 Bandeira Lins              0       2     2     0    
##  9 Camargo Aranha Filho      32     109   141     0.227
## 10 Camilo Léllis             32     133   165     0.194
## # ... with 55 more rows
\end{verbatim}
\end{frame}

\begin{frame}{Exemplo para o ggplot}
\protect\hypertarget{exemplo-para-o-ggplot}{}
\end{frame}

\begin{frame}[fragile]{Unindo e separando colunas}
\protect\hypertarget{unindo-e-separando-colunas}{}
\begin{itemize}
\tightlist
\item
  \texttt{unite} junta duas ou mais colunas usando algum separador
  (\texttt{\_}, por exemplo).
\item
  \texttt{separate} faz o inverso de \texttt{unite}, e uma coluna em
  várias usando um separador.
\end{itemize}
\end{frame}

\begin{frame}{Exemplo de separação de colunas}
\protect\hypertarget{exemplo-de-separauxe7uxe3o-de-colunas}{}
\begin{itemize}
\tightlist
\item
  Olhe os valores da variável classe\_assunto
\end{itemize}
\end{frame}

\begin{frame}[fragile]{Exemplo de separação de colunas}
\protect\hypertarget{exemplo-de-separauxe7uxe3o-de-colunas-1}{}
\begin{itemize}
\item
  Vamos separar a coluna classe\_assunto em duas colunas
\item
  coluna classe e coluna assunto
\item
  Existe separador? -\textgreater{} sim, \texttt{/}
\item
  Usei count apenas em assunto
\end{itemize}
\end{frame}

\begin{frame}[fragile]{Em ação}
\protect\hypertarget{em-auxe7uxe3o-4}{}
\begin{Shaded}
\begin{Highlighting}[]
\NormalTok{decisoes }\OperatorTok{\%\textgreater{}\%}\StringTok{ }
\StringTok{  }\KeywordTok{select}\NormalTok{(n\_processo, classe\_assunto) }\OperatorTok{\%\textgreater{}\%}\StringTok{ }
\StringTok{  }\KeywordTok{separate}\NormalTok{(classe\_assunto, }\KeywordTok{c}\NormalTok{(}\StringTok{\textquotesingle{}classe\textquotesingle{}}\NormalTok{, }\StringTok{\textquotesingle{}assunto\textquotesingle{}}\NormalTok{), }\DataTypeTok{sep =} \StringTok{\textquotesingle{} / \textquotesingle{}}\NormalTok{, }
           \DataTypeTok{extra =} \StringTok{\textquotesingle{}merge\textquotesingle{}}\NormalTok{, }\DataTypeTok{fill =} \StringTok{\textquotesingle{}right\textquotesingle{}}\NormalTok{) }\OperatorTok{\%\textgreater{}\%}\StringTok{ }
\StringTok{  }\KeywordTok{count}\NormalTok{(assunto, }\DataTypeTok{sort =} \OtherTok{TRUE}\NormalTok{)}

\CommentTok{\#\# count é um jeito resumido de usar group\_by() \%\textgreater{}\% summarise(n())}
\end{Highlighting}
\end{Shaded}
\end{frame}

\begin{frame}[fragile]{Em ação}
\protect\hypertarget{em-auxe7uxe3o-5}{}
\begin{verbatim}
## # A tibble: 152 x 2
##    assunto                                n
##    <chr>                              <int>
##  1 Tráfico de Drogas e Condutas Afins  2441
##  2 Pena Privativa de Liberdade         1106
##  3 Roubo Majorado                      1093
##  4 Furto Qualificado                    838
##  5 Roubo                                780
##  6 Progressão de Regime                 607
##  7 Furto                                450
##  8 Receptação                           353
##  9 Homicídio Qualificado                329
## 10 Crimes de Trânsito                   322
## # ... with 142 more rows
\end{verbatim}
\end{frame}

\begin{frame}[fragile]{List columns: \texttt{nest()} e
\texttt{unnest()}}
\protect\hypertarget{list-columns-nest-e-unnest}{}
\texttt{nest()} e \texttt{unnest()} são operações inversas e servem para
tratar dados complexos, como o que temos em \texttt{processos}

\begin{Shaded}
\begin{Highlighting}[]
\NormalTok{d\_partes \textless{}{-}}\StringTok{ }\NormalTok{processos }\OperatorTok{\%\textgreater{}\%}\StringTok{ }
\StringTok{  }\KeywordTok{select}\NormalTok{(n\_processo, partes) }\OperatorTok{\%\textgreater{}\%}\StringTok{ }
\StringTok{  }\KeywordTok{unnest}\NormalTok{(partes)}
\end{Highlighting}
\end{Shaded}
\end{frame}

\begin{frame}[fragile]{}
\protect\hypertarget{section-7}{}
As list columns são uma forma condensada de guardar dados que estariam
em múltiplas tabelas. Por exemplo, uma alternativa à colocar as
\texttt{partes} numa list column seria guardar a tabela
\texttt{d\_partes} separadamente.

\begin{Shaded}
\begin{Highlighting}[]
\KeywordTok{glimpse}\NormalTok{(d\_partes)}
\end{Highlighting}
\end{Shaded}

\begin{verbatim}
## Rows: 37,579
## Columns: 5
## $ n_processo <chr> "0000003-71.2016.8.26.0073", "0000003-71.2016.8.26.0073"...
## $ id         <int> 1, 1, 2, 1, 1, 2, 1, 1, 2, 1, 1, 2, 1, 1, 2, 1, 1, 2, 1,...
## $ name       <chr> "JOSE MARIA JUSTINO NETO", "Defensoria Pública do Estado...
## $ part       <chr> "Apelante", "Apelante", "Apelado", "Apelante", "Apelante...
## $ role       <chr> "Apelante", "Apelante", "Apelado", "Apelante", "Advogado...
\end{verbatim}
\end{frame}

\begin{frame}[fragile]{Duplicatas}
\protect\hypertarget{duplicatas}{}
Para retirar duplicatas, utilizar \texttt{distinct}. Ele considera
apenas a primeira linha em que encontra um padrão para as combinações de
variáveis escolhidas e descarta as demais.

\begin{Shaded}
\begin{Highlighting}[]
\NormalTok{decisoes }\OperatorTok{\%\textgreater{}\%}\StringTok{ }
\StringTok{  }\KeywordTok{distinct}\NormalTok{(municipio)}
\end{Highlighting}
\end{Shaded}

\begin{verbatim}
## # A tibble: 315 x 1
##    municipio          
##    <chr>              
##  1 Cosmópolis         
##  2 São Paulo          
##  3 Ribeirão Preto     
##  4 Araçatuba          
##  5 Presidente Prudente
##  6 Bertioga           
##  7 Taubaté            
##  8 Aparecida          
##  9 Jandira            
## 10 Flórida Paulista   
## # ... with 305 more rows
\end{verbatim}
\end{frame}

\begin{frame}[fragile]{Por coluna}
\protect\hypertarget{por-coluna}{}
Para manter as demais colunas, use \texttt{.keep\_all=}:

\begin{Shaded}
\begin{Highlighting}[]
\NormalTok{decisoes }\OperatorTok{\%\textgreater{}\%}
\StringTok{  }\KeywordTok{distinct}\NormalTok{(municipio, camara, }
           \DataTypeTok{.keep\_all =} \OtherTok{TRUE}\NormalTok{)}
\end{Highlighting}
\end{Shaded}

\begin{verbatim}
## # A tibble: 2,760 x 9
##    id_decisao n_processo classe_assunto municipio camara data_decisao
##    <chr>      <chr>      <chr>          <chr>     <chr>  <chr>       
##  1 11094999   0057003-2~ Habeas Corpus~ Cosmópol~ 3ª Câ~ 19/12/2017  
##  2 11093733   0052762-0~ Habeas Corpus~ São Paulo 3ª Câ~ 19/12/2017  
##  3 11093677   0055169-7~ Habeas Corpus~ Ribeirão~ 3ª Câ~ 19/12/2017  
##  4 11093270   9000580-8~ Agravo de Exe~ Araçatuba 8ª Câ~ 14/12/2017  
##  5 11093374   0052938-7~ Mandado de Se~ São Paulo 8ª Câ~ 14/12/2017  
##  6 11093320   9000723-7~ Agravo de Exe~ Presiden~ 8ª Câ~ 14/12/2017  
##  7 11091506   0003276-8~ Apelação / Tr~ Bertioga  8ª Câ~ 14/12/2017  
##  8 11093326   9000298-1~ Agravo de Exe~ Taubaté   8ª Câ~ 14/12/2017  
##  9 11092475   0004653-3~ Apelação / Tr~ Aparecida 8ª Câ~ 14/12/2017  
## 10 11093773   2221930-6~ Habeas Corpus~ Jandira   3ª Câ~ 19/12/2017  
## # ... with 2,750 more rows, and 3 more variables: data_registro <chr>,
## #   juiz <chr>, txt_decisao <chr>
\end{verbatim}
\end{frame}

\begin{frame}[fragile]{\texttt{janitor::get\_dupes()}}
\protect\hypertarget{janitorget_dupes}{}
Use \texttt{janitor::get\_dupes()} para averiguar os casos em que há
repetição de combinações de colunas.

\begin{Shaded}
\begin{Highlighting}[]
\NormalTok{decisoes }\OperatorTok{\%\textgreater{}\%}\StringTok{ }
\StringTok{  }\KeywordTok{get\_dupes}\NormalTok{(n\_processo)}
\end{Highlighting}
\end{Shaded}

\begin{verbatim}
## # A tibble: 114 x 10
##    n_processo dupe_count id_decisao classe_assunto municipio camara data_decisao
##    <chr>           <int> <chr>      <chr>          <chr>     <chr>  <chr>       
##  1 0000276-8~          2 11051087   Apelação / Tr~ Itaporan~ 2ª Câ~ 27/11/2017  
##  2 0000276-8~          2 11093633   Embargos de D~ Itaporan~ 2ª Câ~ 18/12/2017  
##  3 0000358-1~          2 11108278   Embargos de D~ São Paulo 2ª Câ~ 15/01/2018  
##  4 0000358-1~          2 11028129   Apelação / Ro~ São Paulo 2ª Câ~ 27/11/2017  
##  5 0002236-1~          2 11041351   Apelação / Co~ Nhandeara 1ª Câ~ 27/11/2017  
##  6 0002236-1~          2 11041352   Apelação / Co~ Nhandeara 1ª Câ~ 27/11/2017  
##  7 0004453-2~          2 11041132   Apelação / Tr~ São Paulo 2ª Câ~ 13/11/2017  
##  8 0004453-2~          2 11093635   Embargos de D~ São Paulo 2ª Câ~ 18/12/2017  
##  9 0004636-5~          3 11032094   Apelação / Tr~ Olímpia   2ª Câ~ 27/11/2017  
## 10 0004636-5~          3 11032093   Apelação / Tr~ Olímpia   2ª Câ~ 27/11/2017  
## # ... with 104 more rows, and 3 more variables: data_registro <chr>,
## #   juiz <chr>, txt_decisao <chr>
\end{verbatim}
\end{frame}

\hypertarget{joins}{%
\section{Joins}\label{joins}}

\begin{frame}{Dados relacionais}
\protect\hypertarget{dados-relacionais}{}
\begin{itemize}
\tightlist
\item
  Hadley Wickham \url{http://r4ds.had.co.nz/relational-data.html}
\end{itemize}
\end{frame}

\begin{frame}[fragile]{Principais funções}
\protect\hypertarget{principais-funuxe7uxf5es}{}
Para juntar tabelas, usar \texttt{inner\_join}, \texttt{left\_join},
\texttt{anti\_join}, etc.
\end{frame}

\begin{frame}{Visualizando}
\protect\hypertarget{visualizando}{}
\end{frame}

\begin{frame}[fragile]{Exemplo de inner join:}
\protect\hypertarget{exemplo-de-inner-join}{}
\begin{Shaded}
\begin{Highlighting}[]
\NormalTok{decisoes }\OperatorTok{\%\textgreater{}\%}\StringTok{ }
\StringTok{  }\KeywordTok{filter}\NormalTok{(data\_registro }\OperatorTok{==}\StringTok{ "18/01/2018"}\NormalTok{, }\OperatorTok{!}\KeywordTok{is.na}\NormalTok{(id\_decisao)) }\OperatorTok{\%\textgreater{}\%}\StringTok{ }
\StringTok{  }\KeywordTok{select}\NormalTok{(id\_decisao, n\_processo) }\OperatorTok{\%\textgreater{}\%}\StringTok{ }
\StringTok{  }\KeywordTok{inner\_join}\NormalTok{(processos, }\StringTok{"n\_processo"}\NormalTok{)}
\end{Highlighting}
\end{Shaded}
\end{frame}

\begin{frame}[fragile]{}
\protect\hypertarget{section-8}{}
\begin{verbatim}
## # A tibble: 169 x 5
##    id_decisao n_processo             infos           partes        results      
##    <chr>      <chr>                  <list>          <list>        <list>       
##  1 11109089   0003779-93.2015.8.26.~ <tibble [14 x ~ <tibble [3 x~ <tibble [1 x~
##  2 11109088   3001293-25.2013.8.26.~ <tibble [13 x ~ <tibble [3 x~ <tibble [1 x~
##  3 11108246   0063566-45.2015.8.26.~ <tibble [14 x ~ <tibble [3 x~ <tibble [1 x~
##  4 11108245   0003528-84.2015.8.26.~ <tibble [14 x ~ <tibble [3 x~ <tibble [1 x~
##  5 11109087   0008470-76.2015.8.26.~ <tibble [14 x ~ <tibble [4 x~ <tibble [1 x~
##  6 11109086   0013767-62.2012.8.26.~ <tibble [14 x ~ <tibble [4 x~ <tibble [1 x~
##  7 11109085   3019561-54.2013.8.26.~ <tibble [14 x ~ <tibble [3 x~ <tibble [1 x~
##  8 11108348   0003072-91.2017.8.26.~ <tibble [11 x ~ <tibble [3 x~ <tibble [1 x~
##  9 11108725   0009578-41.2017.8.26.~ <tibble [12 x ~ <tibble [4 x~ <tibble [1 x~
## 10 11108347   3001116-52.2013.8.26.~ <tibble [12 x ~ <tibble [3 x~ <tibble [1 x~
## # ... with 159 more rows
\end{verbatim}
\end{frame}

\begin{frame}[fragile]{Exemplo de right join:}
\protect\hypertarget{exemplo-de-right-join}{}
\begin{Shaded}
\begin{Highlighting}[]
\NormalTok{decisoes }\OperatorTok{\%\textgreater{}\%}\StringTok{ }
\StringTok{  }\KeywordTok{filter}\NormalTok{(data\_registro }\OperatorTok{==}\StringTok{ "18/01/2018"}\NormalTok{, }\OperatorTok{!}\KeywordTok{is.na}\NormalTok{(id\_decisao)) }\OperatorTok{\%\textgreater{}\%}\StringTok{ }
\StringTok{  }\KeywordTok{select}\NormalTok{(id\_decisao, n\_processo) }\OperatorTok{\%\textgreater{}\%}\StringTok{ }
\StringTok{  }\KeywordTok{right\_join}\NormalTok{(processos, }\StringTok{"n\_processo"}\NormalTok{)}
\end{Highlighting}
\end{Shaded}
\end{frame}

\begin{frame}[fragile]{}
\protect\hypertarget{section-9}{}
\begin{verbatim}
## # A tibble: 11,638 x 5
##    id_decisao n_processo             infos           partes        results      
##    <chr>      <chr>                  <list>          <list>        <list>       
##  1 11109089   0003779-93.2015.8.26.~ <tibble [14 x ~ <tibble [3 x~ <tibble [1 x~
##  2 11109088   3001293-25.2013.8.26.~ <tibble [13 x ~ <tibble [3 x~ <tibble [1 x~
##  3 11108246   0063566-45.2015.8.26.~ <tibble [14 x ~ <tibble [3 x~ <tibble [1 x~
##  4 11108245   0003528-84.2015.8.26.~ <tibble [14 x ~ <tibble [3 x~ <tibble [1 x~
##  5 11109087   0008470-76.2015.8.26.~ <tibble [14 x ~ <tibble [4 x~ <tibble [1 x~
##  6 11109086   0013767-62.2012.8.26.~ <tibble [14 x ~ <tibble [4 x~ <tibble [1 x~
##  7 11109085   3019561-54.2013.8.26.~ <tibble [14 x ~ <tibble [3 x~ <tibble [1 x~
##  8 11108348   0003072-91.2017.8.26.~ <tibble [11 x ~ <tibble [3 x~ <tibble [1 x~
##  9 11108725   0009578-41.2017.8.26.~ <tibble [12 x ~ <tibble [4 x~ <tibble [1 x~
## 10 11108347   3001116-52.2013.8.26.~ <tibble [12 x ~ <tibble [3 x~ <tibble [1 x~
## # ... with 11,628 more rows
\end{verbatim}
\end{frame}

\hypertarget{limpeza}{%
\section{Limpeza}\label{limpeza}}

\begin{frame}[fragile]{Duplicatas}
\protect\hypertarget{duplicatas-1}{}
Para retirar duplicatas, utilizar \texttt{distinct}. Ele considera
apenas a primeira linha em que encontra um padrão para as combinações de
variáveis escolhidas e descarta as demais.

\begin{Shaded}
\begin{Highlighting}[]
\NormalTok{decisoes }\OperatorTok{\%\textgreater{}\%}\StringTok{ }
\StringTok{  }\KeywordTok{distinct}\NormalTok{(municipio)}
\end{Highlighting}
\end{Shaded}
\end{frame}

\begin{frame}[fragile]{}
\protect\hypertarget{section-10}{}
\begin{verbatim}
## # A tibble: 315 x 1
##    municipio          
##    <chr>              
##  1 Cosmópolis         
##  2 São Paulo          
##  3 Ribeirão Preto     
##  4 Araçatuba          
##  5 Presidente Prudente
##  6 Bertioga           
##  7 Taubaté            
##  8 Aparecida          
##  9 Jandira            
## 10 Flórida Paulista   
## # ... with 305 more rows
\end{verbatim}
\end{frame}

\begin{frame}[fragile]{Por coluna}
\protect\hypertarget{por-coluna-1}{}
Para manter as demais colunas, use \texttt{.keep\_all=}:

\begin{Shaded}
\begin{Highlighting}[]
\NormalTok{decisoes }\OperatorTok{\%\textgreater{}\%}
\StringTok{  }\KeywordTok{distinct}\NormalTok{(municipio, camara, }
           \DataTypeTok{.keep\_all =} \OtherTok{TRUE}\NormalTok{)}
\end{Highlighting}
\end{Shaded}
\end{frame}

\begin{frame}[fragile]{}
\protect\hypertarget{section-11}{}
\begin{verbatim}
## # A tibble: 2,760 x 9
##    id_decisao n_processo classe_assunto municipio camara data_decisao
##    <chr>      <chr>      <chr>          <chr>     <chr>  <chr>       
##  1 11094999   0057003-2~ Habeas Corpus~ Cosmópol~ 3ª Câ~ 19/12/2017  
##  2 11093733   0052762-0~ Habeas Corpus~ São Paulo 3ª Câ~ 19/12/2017  
##  3 11093677   0055169-7~ Habeas Corpus~ Ribeirão~ 3ª Câ~ 19/12/2017  
##  4 11093270   9000580-8~ Agravo de Exe~ Araçatuba 8ª Câ~ 14/12/2017  
##  5 11093374   0052938-7~ Mandado de Se~ São Paulo 8ª Câ~ 14/12/2017  
##  6 11093320   9000723-7~ Agravo de Exe~ Presiden~ 8ª Câ~ 14/12/2017  
##  7 11091506   0003276-8~ Apelação / Tr~ Bertioga  8ª Câ~ 14/12/2017  
##  8 11093326   9000298-1~ Agravo de Exe~ Taubaté   8ª Câ~ 14/12/2017  
##  9 11092475   0004653-3~ Apelação / Tr~ Aparecida 8ª Câ~ 14/12/2017  
## 10 11093773   2221930-6~ Habeas Corpus~ Jandira   3ª Câ~ 19/12/2017  
## # ... with 2,750 more rows, and 3 more variables: data_registro <chr>,
## #   juiz <chr>, txt_decisao <chr>
\end{verbatim}
\end{frame}

\begin{frame}[fragile]{\texttt{janitor::get\_dupes()}}
\protect\hypertarget{janitorget_dupes-1}{}
Use \texttt{janitor::get\_dupes()} para averiguar os casos em que há
repetição de combinações de colunas.

\begin{Shaded}
\begin{Highlighting}[]
\NormalTok{decisoes }\OperatorTok{\%\textgreater{}\%}\StringTok{ }
\StringTok{  }\KeywordTok{get\_dupes}\NormalTok{(n\_processo)}
\end{Highlighting}
\end{Shaded}
\end{frame}

\begin{frame}[fragile]{}
\protect\hypertarget{section-12}{}
\begin{verbatim}
## # A tibble: 114 x 10
##    n_processo dupe_count id_decisao classe_assunto municipio camara data_decisao
##    <chr>           <int> <chr>      <chr>          <chr>     <chr>  <chr>       
##  1 0000276-8~          2 11051087   Apelação / Tr~ Itaporan~ 2ª Câ~ 27/11/2017  
##  2 0000276-8~          2 11093633   Embargos de D~ Itaporan~ 2ª Câ~ 18/12/2017  
##  3 0000358-1~          2 11108278   Embargos de D~ São Paulo 2ª Câ~ 15/01/2018  
##  4 0000358-1~          2 11028129   Apelação / Ro~ São Paulo 2ª Câ~ 27/11/2017  
##  5 0002236-1~          2 11041351   Apelação / Co~ Nhandeara 1ª Câ~ 27/11/2017  
##  6 0002236-1~          2 11041352   Apelação / Co~ Nhandeara 1ª Câ~ 27/11/2017  
##  7 0004453-2~          2 11041132   Apelação / Tr~ São Paulo 2ª Câ~ 13/11/2017  
##  8 0004453-2~          2 11093635   Embargos de D~ São Paulo 2ª Câ~ 18/12/2017  
##  9 0004636-5~          3 11032094   Apelação / Tr~ Olímpia   2ª Câ~ 27/11/2017  
## 10 0004636-5~          3 11032093   Apelação / Tr~ Olímpia   2ª Câ~ 27/11/2017  
## # ... with 104 more rows, and 3 more variables: data_registro <chr>,
## #   juiz <chr>, txt_decisao <chr>
\end{verbatim}
\end{frame}

\begin{frame}[fragile]{}
\protect\hypertarget{section-13}{}
\begin{itemize}
\item
  Janitor exemplos
  \url{http://sfirke.github.io/janitor/articles/janitor.html}
\item
  Missing e imputação
  \url{https://www.analyticsvidhya.com/blog/2016/03/tutorial-powerful-packages-imputing-missing-values/}
\item
  Outliers (critérios, limpeza e gráficos)
\item
  \texttt{stringi} e \texttt{stringr}
\end{itemize}
\end{frame}

\end{document}
